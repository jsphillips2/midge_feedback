

\section*{Results}

The mesocosm experiment revealed a nonlinear relationship between GPP and larval density,
as the first, second, and third-degree terms associated with midge larval density 
were all statistically significant (Table \ref{tab:gpp}). 
This relationship was generally positive, 
although it saturated 
and was possibly negative at the highest densities (Figure \ref{fig:gpp}).
On days 5--6 of the experiment, 
all three sites had similar GPP-density relationships and overall levels of GPP.
However, there were statistically significant day $\times$ site, day $\times$ density,
and site $\times$ density interactions that manifested as differences
between the sites on days 16--18.
For sites E2 and E5, the GPP-density relationship was weaker 
on days 16--18 than on days 5--6, 
while the density effect at E3 remained largely similar through time.
Furthermore, GPP for E3 and E5 was higher on days 16--18 than on days 5--6, 
while for E2 it was lower.
These relationships corrected for the positive effect 
of ambient temperature during the measurement incubations.
Therefore, the temporal patterns likely reflect real divergences 
between the productivity of the mesocosms at the three sites through time, 
rather than transient differences in ambient environmental conditions 
during the measurements.

The proportion of individuals that emerged as adults declined with initial larval density 
(Type II LRT: $\chi^2_{(1)}$=31.9; $\emph{P}$ < 0.001), 
indicating negative density dependence.
Neither the main effect of site
($\chi^2_{(2)}$=2.1; $\emph{P}$ = 0.422)
nor its interaction with density 
($\chi^2_{(2)}$=4.5; $\emph{P}$ = 0.147) were statistically significant.
Proportional emergence increased with the GPP per initial larva 
(LRT: $\chi^2_{(2)}$=29.8; $\emph{P}$ < 0.001; Figure \ref{fig:adults}),
which is consistent with the hypothesis that negative density dependence is 
related in part to food limitation.

We assessed the effect of larval density on adult emergence 
by projecting the number of emerging midges under two scenarios: 
(i) including the effect of larval density on overall GPP
and (ii) assuming that GPP was constant and therefore was not affected by midge larvae.  
Differences between the two scenarios quantified the consequences of midge effects on GPP
for density-dependent emergence.
The positive effect of larvae on GPP
reduced the negative effect of larval density
on the proportion of individuals that emerged as adults
(Figure \ref{fig:feed}a).
However, because the midge effect on GPP plateaued at high initial midge density 
(Figure \ref{fig:gpp}),
proportional emergence at high density converged on what it would be 
without the midge effect on GPP.
There was modest variation in the midge effect 
on density dependence among the three sites, with the effect being greatest at E3. 
This reflects the fact that the positive midge effect 
on GPP declined through time at E2 and E5, 
while at E3 it remained relatively constant. 
The nonlinear effect of midge larvae on GPP resulted in a hump-shaped relationship
between total emergence and larval density (Figure \ref{fig:feed}b),
indicating that the greatest emergence occurred at intermediate densities.
