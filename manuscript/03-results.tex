

\section*{Results}
 
The 1st, 2nd, and 3rd degree terms associated with midge larval density
were all statistically significant (Table \ref{tab:gpp}), 
indicating a nonlinear relationship between GPP and larval density.
This relationship was generally positive, 
although it saturated 
and was possibly negative at the highest densities (Figure \ref{fig:gpp}).
On day 7 of the experiment, 
all three sites had similar GPP-midge relationships and overall levels of GPP.
However, there were significant day $\times$ site, day $\times$ density,
and site $\times$ density interactions that manifested as a differences
between the sites on day 20.
In particular, the GPP-midge relationship was weaker on day 20 than on day 7 for sites
E2 and E5, while the midge effect at E3 remained largely similar.
The sites also diverged in overall GPP through time,
with E3 and E5 higher on day 20 than on day 7,
while E2 was lower.
These relationships corrected for the significantly positive effect 
of ambient temperature during the measurement incubations.
Therefore, the temporal patterns likely reflect real divergences 
between the productivity of the mesocosms at the three sites through time, 
rather than transient differences in ambient environmental conditions during the measurements.

The emergence rates of adults declined with initial larval density 
(Table \ref{tab:adult}), 
indicating negative density dependence.
Neither the main effect of site nor its interaction with density was significant.
Emergence rates increased with the GPP per initial larva 
(LRT: $\chi^2_{(1)}$=29.8; $\emph{P}$ < 0.001; Figure \ref{fig:adults}),
which is consistent with the hypothesis that negative density dependence is 
related in part to food limitation.
Accordingly, the positive effect of larval density on GPP reduced the strength of 
negative density dependence across the range of densities 
used in the experiment (Figure \ref{fig:feed}).
The positive effect of midges on their own emergence was maximized at intermediate
densities, 
which corresponds to where the effect of larval density on GPP was maximized 
(Figure \ref{fig:gpp}).
Above this density, 
the midge effect on GPP plateaued and perhaps even became slightly negative. 
Therefore, at the highest densities emergence rates converged on what they would be 
in the absence of midge effects on GPP. 
There was modest variation in the midge effect on density dependence among the three sites,
with the effect being greatest at E3.
This reflects the fact that the positive midge effect on GPP declined through time
at sites E2 and E5, while at E3 it remained largely consistent. 

