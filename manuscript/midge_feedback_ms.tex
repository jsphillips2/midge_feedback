% ---------------------------------------------------------------------------------------
% ---------------------------------------------------------------------------------------
% Specifications
% ---------------------------------------------------------------------------------------
% ---------------------------------------------------------------------------------------

\documentclass[12pt]{article}
% \usepackage[sc]{mathpazo} % Like Palatino with extensive math support
\usepackage[letterpaper, margin=1in]{geometry}
% \usepackage{mathptmx} % Like Times New Roman
\usepackage{newtxtext,newtxmath}
\usepackage[math-style=ISO]{unicode-math}
\usepackage{fullpage}
\usepackage[authoryear,sectionbib,sort]{natbib}

\linespread{1.7}

\usepackage[utf8]{inputenc}
\usepackage{lineno}
\usepackage{titlesec}
\titleformat{\section}[block]{\Large\bfseries\filcenter}{\thesection}{1em}{}
\titleformat{\subsection}[block]{\Large\itshape\filcenter}{\thesubsection}{1em}{}
\titleformat{\subsubsection}[block]{\large\itshape}{\thesubsubsection}{1em}{}
\titleformat{\paragraph}[runin]{\itshape}{\theparagraph}{1em}{}[. ]\renewcommand{\refname}{Literature Cited}

% For Icelandic ð symbol:
\DeclareTextSymbolDefault{\dh}{T1}
% Increased spacing in math mode:
\medmuskip=8mu % by default it is equal to 4 mu
\thickmuskip=10mu % by default it is equal to 5 mu

% Figures
\usepackage{graphicx}
% Table
\usepackage{booktabs}
\renewcommand{\thetable}{\Roman{table}}






% ---------------------------------------------------------------------------------------
% ---------------------------------------------------------------------------------------
% Title page
% ---------------------------------------------------------------------------------------
% ---------------------------------------------------------------------------------------

\title{Ecosystem engineering weakens negative density dependence in an aquatic insect population}

\author{
Joseph S. Phillips$^{1,2,\dagger}$ \\
Amanda R. McCormick$^{1}$ \\
Jamieson C. Botsch$^{1}$ \\
Anthony R. Ives$^{1}$}

\usepackage{amsmath} % for split math environment

\date{}

\begin{document}

\raggedright
\setlength\parindent{0.25in}

\maketitle

\noindent{} 1. Department of Integrative Biology, University of Wisconsin, Madison, Wisconsin 53706 USA

\noindent{} 2. Department of Aquaculture and Fish Biology, H\'{o}lar University, Skagafj\"{o}r{\dh}ur 551 Iceland

\noindent{} $\dagger$ E-mail: joseph@holar.is

\bigskip

Running head: {Ecosystem engineering feedbacks}

\linenumbers{}

\clearpage





% ---------------------------------------------------------------------------------------
% ---------------------------------------------------------------------------------------
% Abstract
% ---------------------------------------------------------------------------------------
% ---------------------------------------------------------------------------------------


\section*{Abstract}



\bigskip

\textit{Keywords}: {benthic production; 
                    facilitation; 
                    feedbacks; 
                    interpsecific interactions; 
                    macroinvertebrates;
                    \emph{T. gracilentus}}

\clearpage





% ---------------------------------------------------------------------------------------
% ---------------------------------------------------------------------------------------
% Introduction
% ---------------------------------------------------------------------------------------
% ---------------------------------------------------------------------------------------

\section*{Introduction}

Ecosystem engineering is a class of ecological interactions whereby effects of one
population on another are mediated through alterations to the physical environment
\citep{jones1994, wilby2002}.
Like all interspecific interactions, 
ecosystem engineering has the potential to generate feedbacks among various 
members of a community \citep{largaespada2012,donadi2014,sanders2014}.
For example, physical structure provided by coral can ameliorate competition with algae by
benefiting grazers that reduce algal abundance \citep{bozec2013}. 
As ecosystem engineers are (by definition) the source of engineering effects within
an ecosystem, 
feedbacks between engineering and the engineers themselves
are central to the dynamical consequences of engineering for the 
community as a whole \citep{hastings2007, sanders2014}. 

To understand the role of engineering feedbacks
for the population dynamics of ecosystem engineers,
it is essential to relate those feedbacks to the strength of density dependence
\citep{hastings2007, cuddington2009}.
Using a simple mathematical model,
\cite{cuddington2009} showed that a wide range of dynamical behavior
is possible for populations of ecosystem engineers,
including stable persistence, extinction, unbounded growth, and alternative states.
Two key factors for determining such outcomes are 
(1) the dependence of engineering effect on population density and 
(2) the feedback of engineering to density-dependence.
Despite their theoretical importance,
quantitative characterizations of such relationships for natural populations are limited.
While previous studies have established the existence of engineering-mediated feedbacks
to engineering populations \citep[e.g.,][]{bozec2013, donadi2014, largaespada2012}
they have generally not done so across a range of engineers densities 
as is required to directly quantify density dependence.
 
We quantified the effect of ecosystem engineering on the strength of sign and magnitude
of density dependence in the midge \emph{Tanytarsus gracilenuts} (Diptera: Chironomidae)
in Lake M\'{y}vatn, Iceland. 
The larvae of \emph{T. gracilenuts} dwell in the sediment and build silk tubes that 
elevate primary production by providing a substrate for algal growth 
\citep{herren2017, phillips2019},
similar to other aquatic macroinvertbrates 
\citep{largaespada2012,donadi2014,hoelker2015}.
The larva feed on benthic algae, 
which means that their enhancement of benthic production may benefit their own 
survival and subsequent reproduction \citep{ingvason2004, einarsson2002}.
However, midge consumption may also reduce algal biomass,
potentially leading to intraspecific competition \citep{einarsson2016}.
Indeed, M\'{y}vatn's \emph{T. gracilenuts} show large fluctuations 
in abundance that are likely driven by food limiation,
although these fluctuations cannot be explained purely in term classical 
consumer-resource cylces \citep{ives2008}.
Characterizing the nature of density dependence is important for understanding 
the complex population dynamics of \emph{T. gracilenuts},
making it a good example case for exploring the effects of ecosytem engineering 
on density dependence.

To evaluate the role of ecosystem engineering on density dependence in \emph{T. gracilenuts},
we conducted a field mesocosm experiment across a range of experimental larval densties. 
This allowed us to directly quantify
(a) the relationship between benthic primary produdction and larval midge density and
(b) the relationship between adult emergence rates and primary produdction per larval midge.
We then combined these two relationships into a simple model 
that allowed to isolate the contribution of larval midge effects on 
primary production to their density dependence.






% ---------------------------------------------------------------------------------------
% ---------------------------------------------------------------------------------------
% Methods
% ---------------------------------------------------------------------------------------
% ---------------------------------------------------------------------------------------




\section*{Methods}


M\'{y}vatn is a large ($37\text{m}^2$), shallow (mean depth: 2.5m), 
naturally eutrophic lake in northeastern Iceland (65°40’N 17°00’W) \citep{einarsson2004}.
It is separated into two, ecologically distinct basins (north and south).
Our study was conducted in 2017 at three sites (E2, E3, and E5) in the south basin 
that were selected to represent a range of ecological conditions (Figure \ref{fig:sites}).
These sites all have soft substrate that is suitable for \emph{T. gracilentus},
although they can differ in larval densities. 
In sediment cores taken throughout the summer of 2017, E3 had the highest densities 
(mean $\pm$ standard error: $69,058 \pm 14,595~\text{m}^{-2}$),
followed by E5 ($30,648 \pm 11,767$), and then E2 ($431 \pm 172$).
Maximum densities in M\'{y}vatn have exceeded $200,000~\text{m}^{-2}$ \citep{lindegaard1979}.
In the summer of 2017, 
E2 was subject an expanding mat of filamentous green algae (Cladophorales) 
that was largely absent from E3 and E5 (J. Phillips; personal observation).
Furthermore, E2 was substantially colder during the experiment period than 
E3 and E5 (Figure \ref{fig:sites}).
In contrast, photosynthetically active radiation (PAR) was similar between the sites,
due to their similar depths (E2: 2.8m; E3: 3.3m; E5: 2.6m)
and similar water clarity throughout the south basin in 2017.
Light and temperature data were collected with two loggers 
(HOBO Pendant, Onset Computer Corporation) deployed on the lake bottom at each site
and set to log every 30 minutes.
PAR was recorded as visual light intensity (lux)
and approximately converted to PAR using a standard scaling factor \citep{thimijan1983}.

We conducted our field mesocosm experiment using a design
similar to \citep{phillips2019}.
On 28 June 2017, we collected sediment cores from the three study sites using a Kajak corer. 
For each site, we pooled the sediment from the different cores while keeping the 
top 5cm (``top'') and next 10cm (``bottom'') separate.
We then sieved the sediment through either 125 (top) or 500$\mu \text{m}$ (bottom) mesh
to remove midge larva.
The sediment was left to settle for 4 days in a cool, dark, location.
We constructed the mesocosms by stocking the sediment into 
clear acrylic tubes (33cm height $\times$ 5cm diameter) 
sealed from the bottom with foam stoppers.
We fist added 10cm of bottom sediment and then 5cm of top sediment, 
to mimic the layering in the lake.
The sediment layer of each mesocosm was wrapped with 4 layers of black plastic
to eliminate light from the sides of the mesocosms.

On 3 July, 
we took sediment cores at E3 and sieved them through 125 $\mu \text{m}$
to collect Tanytarsini larvae (the vast majority of which were likely \emph{T. gracilentus}).
Tanytarsini progress through four instars before emerging as adults.
We attempted to select individuals the general size of second instar larvae
to maximize the duration of the experiment before emergence.
The following day we stocked the mesocosms with four densities of Tanytarsini larvae:
0, 50, 100, 200 per mescosm (0, 25000, 51000, and 102000 $\text{m}^{-2}$). 
We then filled the mesocosms with water collected from the southern shore of M\'{y}vatn's
south basin and gave the midges 24 h to settle before deploying in the lake.
On 5 July, 
we distributed the mesocosms corresponding to each site onto two racks and then
deployed them at their respective sites on the lake bottom.
The tops of the mesocosms were left open to allow exchange between the mesocosms
and the lake water column.

On 10 and 11 July, 
we estimated gross primary production (GPP) in the mesocosms by measurnig the change in
dissolved oxygen (DO) concentration during sealed incubations \citep{hall2017}.
The incubations were conducted in situ at the respective sites to incorporate
spatial variation in ambient conditions, such as light and temperature.
Each mesocosm was first incubated under ambient light to give an
estimate of net ecosystem production (NEP),
followed by an incubation under dark conditions produced by wrapping each mesocosm 
in 4 layers of black plastic to give an estimate of ecosystem respiration (ER).
NEP + ER gives an estimate of GPP, assuming that ER is the same during both the light
and dark incubations. Half of the mesocosms at each site were incubated on 10 July,
while the other half were incubated on 11 July;
all of the mesocosms incubated on a given day for a given site were on the same 
experimental rack and so constituted a ``block''.
The incubations lasted between approximately 3 and 5 hours, 
and the tops of the mesocosms were sealed with rubber stoppers for the duration.
DO was measured using a handheld probe (ProODO, YSI, Yellow Springs, Ohio, USA),
and we gentled stirred the water within each mesocosm to homogenize it 
before taking the reading. 
We repeated the incubation procedure on 21 and 23 July.
For logistical reasons, we were unable to perform the incubations at the respective sites.
Therefore, on 21 July all of the mesocosms were moved to a common location on the
southern shore of the south basin (depth $\approx$ 1.7m).
The incubations lasted between approximately 3 and 5 hours,
while the dark incubations lasted between 4 and 10 hours.
While variation in incubation duration was not ideal,
the amount of DO in the dark incubations remained above anoxic conditions 
(minimum DO >10 $\text{mg}^{-1}$). 
We converted GPP to units of $\text{mg}~\text{O}_2~\text{m}^{-2}~\text{h}^{-1}$,
accounting for incubation duration and water column depth within each mesocosm.

On 23 July, shortly prior to when we expected midges to begin emerging from the mesocosms,
we removed the mesocosms from the lake and secured a mesh cover to the top of each
to catch adult midges as they emerged. 
We kept the mesocosms outdoors in mesh tents,
using water baths of cold tap water to moderate the temperature of the mesocosms.
The water baths had a depth of approximately 18cm,
which was sufficient to cover the sediment portion of each mesocosm while leaving the
tops exposed to the air to allow emergence.
Every 1-3 days for the next 13 days, we collected the emerging adults from the mesocosms.
While these were not individually identified, the vast majority appeared to be Tanytarsini.
Furthermore, there was a strong association between the number of Tanytarsini larvae 
stocked in the mesocosms and the number of adults that emerged 
(Spearman rank correlation of 0.82; $\emph{P}$ < 0.0001; including the zero density treatment).

We quantified the relationship between GPP and initial larval density using a 
linear mixed model (LMM). 
The model included initial density (four levels), site (three levels),
incubation period (two levels; either 10-11 or 21-23 July), 
and their two-way interactions as fixed effects.
Because we expected the relationship between GPP and initial density to be nonlinear,
we also included 2nd and 3rd order polynomial terms for initial density 
(without any interactions) in the model.
We choose a third degree polynomial because this gave the same number of parameters 
to estimate as would have been the case if each of the four treatments were treated as discrete
levels (including the intercept). 
The polynomial regression had the advantages of 
(a) allowing us to treat density as a numeric
variable and 
(b) of allowing us to simplify the model by only allowing interactions with the linear
density term.
We accounted for variation in ambient conditions during the incubations by including
linear terms for PAR and temperature estimated for each block at each site.
Finally, we included random effects for experimental rack and mesocosm identity to account
for blocking and repeated measures, respectively.

We used a binomial generalized linear mixed model (GLMM) to analyze variation 
in the proportion of the initial number of midge larvae that emerged as adults from each mesocosm 
(excluding the zero treatment). 
We included initial density (three levels), site (three levels), and their interaction
as fixed effects. 
We included random effects for experimental rack and mesocosm identity to account
for blocking and potential overdispersion, respectively;
the latter was equivalent to assuming the residuals followed a logit-normal-binomial distribution.
To assess the potential for larval midge effects on GPP to alter the strength of density-dependent
emergence, 
we fit a GLMM similar to the one described above, 
but with GPP per initial midge larva as the sole fixed effect. 
We then generated ``predicted'' values of adult emergence rates under two scenarios 
(1) using predicted values of GPP as a function of site and density treatment according
to the polynomial LMM described above, 
and (2) using predicted values of GPP fixed across larval density, but including variation 
across sites, based on the polynomial LMM evaluated at the mean larval density.
In scenario (2), GPP per larva declined across the midge treatments purely due to 
the partitioning of GPP across a greater number of individuals.
Scenario (1) included this decline in GPP due to partitioning, but also included
any positive or negative effects of larval density on GPP itself.
The difference between scenarios (1) and (2) gave a measure of the effect of larval
density on the strength of density dependence as mediated through midge effects on GPP.

Statistical analyses were conducted in R 4.0.0,
using the ``lme4'' package to fit the LMM and GLMMs.
We calculated $\emph{P}$-values with $\emph{F}$-tests using
the Kenward-Roger correction for the LMM (``Anova'' function in the``car'' package)
and with parametric-bootstrapped likelihood-ratio tests (LRTs) based on 2000 simulations
for the GLMMs (``simulate'' function in the native ``stats'' package).
We used both Type III and Type II tests,
to balance concerns with inflated Type-I errors that can occur when dropping 
terms with the poor statistical inference than can come from overparameterized models.













% ---------------------------------------------------------------------------------------
% ---------------------------------------------------------------------------------------
% Results
% ---------------------------------------------------------------------------------------
% ---------------------------------------------------------------------------------------



\section*{Results}

The mesocosm experiment revealed a nonlinear relationship between GPP and larval density,
as the first, second, and third-degree terms associated with midge larval density 
were all statistically significant (Table \ref{tab:gpp}). 
This relationship was generally positive, 
although it saturated 
and was possibly negative at the highest densities (Figure \ref{fig:gpp}).
On days 5--6 of the experiment, 
all three sites had similar GPP-density relationships and overall levels of GPP.
However, there were statistically significant day $\times$ site, day $\times$ density,
and site $\times$ density interactions that manifested as differences
between the sites on days 16--18.
For sites E2 and E5, the GPP-density relationship was weaker 
on days 16--18 than on days 5--6, 
while the density effect at E3 remained largely similar through time.
Furthermore, GPP for E3 and E5 was higher on days 16--18 than on days 5--6, 
while for E2 it was lower.
These relationships corrected for the positive effect 
of ambient temperature during the measurement incubations.
Therefore, the temporal patterns likely reflect real divergences 
between the productivity of the mesocosms at the three sites through time, 
rather than transient differences in ambient environmental conditions 
during the measurements.

The proportion of individuals that emerged as adults declined with initial larval density 
(Type II LRT: $\chi^2_{(1)}$=31.9; $\emph{P}$ < 0.001), 
indicating negative density dependence.
Neither the main effect of site
($\chi^2_{(2)}$=2.1; $\emph{P}$ = 0.422)
nor its interaction with density 
($\chi^2_{(2)}$=4.5; $\emph{P}$ = 0.147) were statistically significant.
Proportional emergence increased with the GPP per initial larva 
(LRT: $\chi^2_{(2)}$=29.8; $\emph{P}$ < 0.001; Figure \ref{fig:adults}),
which is consistent with the hypothesis that negative density dependence is 
related in part to food limitation.

We assessed the effect of larval density on adult emergence 
by projecting the number of emerging midges under two scenarios: 
(i) including the effect of larval density on overall GPP
and (ii) assuming that GPP was constant and therefore was not affected by midge larvae.  
Differences between the two scenarios quantified the consequences of midge effects on GPP
for density-dependent emergence.
The positive effect of larvae on GPP
reduced the negative effect of larval density
on the proportion of individuals that emerged as adults
(Figure \ref{fig:feed}a).
However, because the midge effect on GPP plateaued at high initial midge density 
(Figure \ref{fig:gpp}),
proportional emergence at high density converged on what it would be 
without the midge effect on GPP.
There was modest variation in the midge effect 
on density dependence among the three sites, with the effect being greatest at E3. 
This reflects the fact that the positive midge effect 
on GPP declined through time at E2 and E5, 
while at E3 it remained relatively constant. 
The nonlinear effect of midge larvae on GPP resulted in a hump-shaped relationship
between total emergence and larval density (Figure \ref{fig:feed}b),
indicating that the greatest emergence occurred at intermediate densities.






% ---------------------------------------------------------------------------------------
% ---------------------------------------------------------------------------------------
% Discussion
% ---------------------------------------------------------------------------------------
% ---------------------------------------------------------------------------------------



\section*{Discussion}

Our field mesocosm experiment shows how ecosystem engineering can weaken negative 
density dependence for the population of engineers.
We found that \emph{T. gracilentus} larvae in  M\'{y}vatn have large positive 
effects on benthic primary production,
which previous studies have shown are driven at least in part by physical 
structure provided by the tubes in which the larvae reside 
\citep{hoelker2015, phillips2019}.
Because \emph{T. gracilentus} larvae feed on benthic diatoms \citep{ingvason2004},
simulation of benthic production increased the amount of production per individual,
which in turn weakened negative density-dependence arising from food limitation.


While the effect off midges on benthic production was generally positive in our experiment,
this effect was nonlinear and plateaued between 60,000 and 90,000 
individuals $\text{m}^{-2}$.
Consequently, 
the midge-mediated weakening of density dependence was maximized
at intermediate densities.
Given the suppression of algal biomass through grazing \citep{einarsson2016},
it is likely that the effect of midges on production becomes negative
at the highest densites observed in the lake (>200,000 $\text{m}^{-2}$).
This suggests that while midges weaken negative density dependence at moderate
densities, they may enhance it at the highest densities.
The nonlinearity of ecosytem engineer effects has previously been identified 
as an important factor in governing the effects of engineering on community dynamics 
\citep{bozec2013}.
Despite weakening densitity dependence at low to moderate densities,
midge engineering did not lead to positive densitiy dependence 
\citep[i.e., allee effects;][]{courchamp1999}.
This has important implications for their population dynamics,
as positive density dependence could lead to run-away or overcompensatory growth
\citep{turchin2003, cuddington2009}.

The midge effect on benthic producivity across the three sites 
was similar at the beginning of the experiment,
but diverged through time even after accounting for variation in ambient conditions
during the productivity measurements.
This suggests that there was a legacy of local conditions that affected the 
response of benthic producers to midge engineering.
While our experiment was not directly able to test for such legacies,
a plausible candidate is temperature,
which was consistently lowest at the site with the weakest response to midges.
Various studies have identified the role of environmental variation in mediating 
the strength and sign of ecosystem engineering \citep{wright2006,lathlean2017}.
Enduring legacies of environemtnal mediation are particularly important,
as they may serve to decouple the dynamics of the engineers 
and their community-wide effects
\citep{cuddington2011}.
Such decoupling may in tern alter the nature of density-dependent feedbacks in 
systems with ecosystem engineering \citep{cuddington2009}.




% ---------------------------------------------------------------------------------------
% ---------------------------------------------------------------------------------------
% Acknowledgments
% ---------------------------------------------------------------------------------------
% ---------------------------------------------------------------------------------------

\section*{Acknowledgments}

This work was supported by National Science Foundation grants 
DEB-1052160, DEB-1556208 to Anthony R. Ives,
and Graduate Research Fellowships DGE-1256259.
The M\'{y}vatn Research Station directed by \'{A}rni Einarsson
provided logistical and scientific support.
We thank Kristian Riley Book, Natalie Schmer, Bethany Smith, and Aspen Ward
for assistance with fieldwork.


% ---------------------------------------------------------------------------------------
% ---------------------------------------------------------------------------------------
% Literature Cited
% ---------------------------------------------------------------------------------------
% ---------------------------------------------------------------------------------------



\bibliographystyle{ecology.bst}
\clearpage

\bibliography{refs.bib}

\clearpage





% ---------------------------------------------------------------------------------------
% ---------------------------------------------------------------------------------------
% Tables & Figures
% ---------------------------------------------------------------------------------------
% ---------------------------------------------------------------------------------------

% ---------------------------------------------------------------------------------------
% Table I
% ---------------------------------------------------------------------------------------

\begin{table}
\caption{\label{tab:gpp}
LMM for mesocosm GPP. 
\emph{P}-values are from \emph{F}-tests with the Kenward-Roger correction.
The model included random effects for block ($\upsigma$ = 0.001)
and mesocosm identity ($\upsigma$ = 0.013), 
with residual standard deviation $\upsigma$ = 0.012.}
\setlength{\tabcolsep}{12pt}
\begin{tabular}{lllllll}
\toprule
& & \multicolumn{2}{c}{Type III} & & \multicolumn{2}{c}{Type II} \\
\cmidrule{3-4} \cmidrule{6-7}
Term & & \emph{F}_{(ndf,ddf)} & \emph{P} & & \emph{F}_{(ndf,ddf)} & \emph{P}\\
\midrule
temperature & & 31_{(1,41.2)} & <0.001 & & 31_{(1,41.2)} & <0.001\\

PAR & & 0.001_{(1,38.9)} & 0.981 & & 0.001_{(1,38.9)} & 0.981\\

period & & 0.35_{(1,41.3)} & 0.555 & & 79_{(1,41.3)} & <0.001\\

site & & 0.15_{(2,4.63)} & 0.861 & & 4_{(2,2.62)} & 0.161\\

density & & 12_{(1,37.8)} & 0.002 & & 13_{(1,37)} & <0.001\\

density^2 & & 5.6_{(1,37.4)} & 0.023 & &  & \\

density^3  & & 4.6_{(1,37.1)} & 0.039 & & & \\

period $\times$ site & & 10_{(2,41)} & <0.001 & & & \\

period $\times$ density & & 16_{(1,39.4)} & <0.001 & & & \\

site $\times$ density & & 4.4_{(2,36.6)} & 0.020 & & & \\
\bottomrule
\end{tabular}
\end{table}


\clearpage



% ---------------------------------------------------------------------------------------
% Table II
% ---------------------------------------------------------------------------------------

\begin{table}
\caption{\label{tab:adult}
GLMM for adult emergence rate, as a proportion of the initial number of individuals.
\emph{P}-values are from a bootstrapped likelihood-ratio test, 
calculated from 2000 simulated data sets.
The model included random effects for block ($\upsigma$ = 0.03)
and mesocosm identity ($\upsigma$ = 0.26).}
\setlength{\tabcolsep}{12pt}
\begin{tabular}{lllllll}
\toprule
& & \multicolumn{2}{c}{Type III} & & \multicolumn{2}{c}{Type II} \\
\cmidrule{3-4} \cmidrule{6-7}
Term & & \chi^2_{(df)} & \emph{P} & & \chi^2_{(df)} & \emph{P}\\
\midrule
density & & 0.0_{(1)} & 0.288 & & 31.9_{(1)} & <0.001\\

site & & 4.7_{(2)} & 0.217 & & 2.1_{(2)} & 0.422\\

density $\times$ site & & 4.5_{(2)} & 0.147 & & & \\
\bottomrule
\end{tabular}
\end{table}

\clearpage



% ---------------------------------------------------------------------------------------
% Figure 1
% ---------------------------------------------------------------------------------------

\begin{figure}
\centering
\includegraphics{../analysis/figures/p_sites.pdf}
\caption{\label{fig:sites}
Experimental sites. 
(a) The three sites covered a wide range of M\'{y}vatn's south basin.
Light gray areas indicate water, while white areas indicate land.
(b) Mean daily PAR and temperature were calculated by averaging half-hourly
measurements from two loggers deployed on the lake bottom for each site.
}
\end{figure}

\clearpage



% ---------------------------------------------------------------------------------------
% Figure 2
% ---------------------------------------------------------------------------------------

\begin{figure}
\centering
\includegraphics{../analysis/figures/p_gpp.pdf}
\caption{\label{fig:gpp}
GPP as a function of initial larval density.
The points show the observed data standardized to the mean water temperature
measured during the incubations.
The solid lines show fitted values from the third-degree polynomial LMM.
While the model provides a smooth fit to the data,
we connected the fitted values at the experimental densities with
straight lines to draw attention to discete levels at which the measurements were taken. 
The shaded regions show the standard errors estimated from the 
covariance matrix associated with the model fit.
}
\end{figure}

\clearpage



% ---------------------------------------------------------------------------------------
% Figure 3
% ---------------------------------------------------------------------------------------

\begin{figure}
\centering
\includegraphics{../analysis/figures/p_adults.pdf}
\caption{\label{fig:adults}
Adult emergence rate as a function of GPP per initial midge larva.
The points show the observed data, 
the line shows teh fitted values from a GLMM,
and the shaded regions show the standard errors estimated from the 
covariance matrix associated with the model fit.
}
\end{figure}

\clearpage

% ---------------------------------------------------------------------------------------
% Figure 4
% ---------------------------------------------------------------------------------------

\begin{figure}
\centering
\includegraphics{../analysis/figures/p_feed.pdf}
\caption{\label{fig:feed}
Consequences of midge effects on GPP for negative density-dependence.
The curves are derived from the conjunction of the LMM in Figure \ref{fig:gpp}
and the GLMM in Figure \ref{fig:adults}. 
The solid lines (``with midge effect'') show modeled emergence rates including the modeled
effect of midges on overall GPP,
while the dashed lines (``no midge effect'') exclude midge effects on overall GPP.
The difference between the two lines is a measure of the midge effect on density-dependent
emergence as mediated by midge effects on GPP. 
}
\end{figure}

\clearpage




% ---------------------------------------------------------------------------------------
% ---------------------------------------------------------------------------------------
% Supplement
% ---------------------------------------------------------------------------------------
% ---------------------------------------------------------------------------------------

% \section*{Supplement}

\end{document}
