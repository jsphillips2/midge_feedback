% ---------------------------------------------------------------------------------------
% ---------------------------------------------------------------------------------------
% Specifications
% ---------------------------------------------------------------------------------------
% ---------------------------------------------------------------------------------------

\documentclass[12pt]{article}
% \usepackage[sc]{mathpazo} % Like Palatino with extensive math support
\usepackage[letterpaper, margin=1in]{geometry}
% \usepackage{mathptmx} % Like Times New Roman
\usepackage{newtxtext,newtxmath}
\usepackage[math-style=ISO]{unicode-math}
\usepackage{fullpage}
\usepackage[authoryear,sectionbib,sort]{natbib}

\linespread{1.7}

\usepackage[utf8]{inputenc}
\usepackage{lineno}
\usepackage{titlesec}
\titleformat{\section}[block]{\Large\bfseries\filcenter}{\thesection}{1em}{}
\titleformat{\subsection}[block]{\Large\itshape\filcenter}{\thesubsection}{1em}{}
\titleformat{\subsubsection}[block]{\large\itshape}{\thesubsubsection}{1em}{}
\titleformat{\paragraph}[runin]{\itshape}{\theparagraph}{1em}{}[. ]\renewcommand{\refname}{Literature Cited}

% For Icelandic ð symbol:
\DeclareTextSymbolDefault{\dh}{T1}
% Increased spacing in math mode:
\medmuskip=8mu % by default it is equal to 4 mu
\thickmuskip=10mu % by default it is equal to 5 mu

% Figures
\usepackage{graphicx}
% Table
\usepackage{booktabs}
\renewcommand{\thetable}{\Roman{table}}






% ---------------------------------------------------------------------------------------
% ---------------------------------------------------------------------------------------
% Title page
% ---------------------------------------------------------------------------------------
% ---------------------------------------------------------------------------------------

\title{Ecosystem engineering weakens negative density dependence in an aquatic insect population}

\author{
Joseph S. Phillips$^{1,2,\dagger}$ \\
Amanda R. McCormick$^{1}$ \\
Jamieson C. Botsch$^{1}$ \\
Anthony R. Ives$^{1}$}

\usepackage{amsmath} % for split math environment

\date{}

\begin{document}

\raggedright
\setlength\parindent{0.25in}

\maketitle

\noindent{} 1. Department of Integrative Biology, University of Wisconsin, Madison, Wisconsin 53706 USA

\noindent{} 2. Department of Aquaculture and Fish Biology, H\'{o}lar University, Skagafj\"{o}r{\dh}ur 551 Iceland

\noindent{} $\dagger$ E-mail: joseph@holar.is

\bigskip

Running head: {Ecosystem engineering feedbacks}

\linenumbers{}

\clearpage





% ---------------------------------------------------------------------------------------
% ---------------------------------------------------------------------------------------
% Abstract
% ---------------------------------------------------------------------------------------
% ---------------------------------------------------------------------------------------


\section*{Abstract}

Ecosystem engineers have large impacts on the communities in which they live,
and these impacts may feed back to populations of engineers themselves.
In this study, we assessed the effect of ecosystem engineering 
on density-dependent feedbacks for midges in Lake M\'{y}vatn, Iceland. 
The midge larvae reside in the sediment and build silk tubes that provide 
a substrate for algal growth, thereby elevating benthic primary production.
Benthic algae are in turn the primary food source for the midge larvae,
setting the stage for the effects of engineering to feed back to the midges themselves.
Using a field mesocosm experiment manipulating larval midge densities,
we found a generally positive but nonlinear relationship between
density and benthic production.
Furthermore, 
adult emergence increased with the primary production per midge larva.
By combining these two relationships in a simple model,
we found that the positive effect of midges on benthic production 
weakened the negative density dependence at low to intermediate larval densities.
However, this benefit disappeared at high densities when midge consumption 
of primary producers exceeded their positive effects 
on primary production through ecosystem engineering.
Our results illustrate how ecosystem engineering can alter 
density-dependent feedbacks for engineer populations.




\bigskip

\textit{Keywords}: {benthic production; 
                    facilitation; 
                    feedbacks; 
                    interpsecific interactions; 
                    macroinvertebrates;
                    \emph{T. gracilentus}}

\clearpage





% ---------------------------------------------------------------------------------------
% ---------------------------------------------------------------------------------------
% Introduction
% ---------------------------------------------------------------------------------------
% ---------------------------------------------------------------------------------------

\section*{Introduction}

Ecosystem engineering is a class of ecological interactions whereby effects of one
population on another are mediated through alterations to the physical environment
\citep{jones1994, wilby2002}.
Like all interspecific interactions, 
ecosystem engineering has the potential to generate feedbacks among various 
members of a community \citep{largaespada2012,donadi2014,sanders2014}.
For example, physical structure provided by coral can ameliorate competition with algae by
benefiting grazers that reduce algal abundance \citep{bozec2013}. 
As ecosystem engineers are (by definition) the source of engineering effects within
an ecosystem, 
feedbacks between engineering and the engineers themselves
are central to the dynamical consequences of engineering for the 
community as a whole \citep{hastings2007, sanders2014}. 

To understand the role of engineering feedbacks
for the population dynamics of ecosystem engineers,
it is essential to relate those feedbacks to the strength of density dependence
\citep{hastings2007, cuddington2009}.
Using a simple mathematical model,
\cite{cuddington2009} showed that a wide range of dynamical behavior
is possible for populations of ecosystem engineers,
including stable persistence, extinction, unbounded growth, and alternative states.
Two key factors for determining such outcomes are 
(1) the dependence of engineering effect on population density and 
(2) the feedback of engineering to density-dependence.
Despite their theoretical importance,
quantitative characterizations of such relationships for natural populations are limited.
While previous studies have established the existence of engineering-mediated feedbacks
to engineering populations \citep[e.g.,][]{bozec2013, donadi2014, largaespada2012}
they have generally not done so across a range of engineers densities 
as is required to directly quantify density dependence.
 
We quantified the effect of ecosystem engineering on the strength of sign and magnitude
of density dependence in the midge \emph{Tanytarsus gracilenuts} (Diptera: Chironomidae)
in Lake M\'{y}vatn, Iceland. 
The larvae of \emph{T. gracilenuts} dwell in the sediment and build silk tubes that 
elevate primary production by providing a substrate for algal growth 
\citep{herren2017, phillips2019},
similar to other aquatic macroinvertbrates 
\citep{largaespada2012,donadi2014,hoelker2015}.
The larva feed on benthic algae, 
which means that their enhancement of benthic production may benefit their own 
survival and subsequent reproduction \citep{ingvason2004, einarsson2002}.
However, midge consumption may also reduce algal biomass,
potentially leading to intraspecific competition \citep{einarsson2016}.
Indeed, M\'{y}vatn's \emph{T. gracilenuts} show large fluctuations 
in abundance that are likely driven by food limiation,
although these fluctuations cannot be explained purely in term classical 
consumer-resource cylces \citep{ives2008}.
Characterizing the nature of density dependence is important for understanding 
the complex population dynamics of \emph{T. gracilenuts},
making it a good example case for exploring the effects of ecosytem engineering 
on density dependence.

To evaluate the role of ecosystem engineering on density dependence in \emph{T. gracilenuts},
we conducted a field mesocosm experiment across a range of experimental larval densties. 
This allowed us to directly quantify
(a) the relationship between benthic primary produdction and larval midge density and
(b) the relationship between adult emergence rates and primary produdction per larval midge.
We then combined these two relationships into a simple model 
that allowed to isolate the contribution of larval midge effects on 
primary production to their density dependence.






% ---------------------------------------------------------------------------------------
% ---------------------------------------------------------------------------------------
% Methods
% ---------------------------------------------------------------------------------------
% ---------------------------------------------------------------------------------------




\section*{Methods}

%========================================================================================

\subsection*{Study system}


%========================================================================================

\subsection*{Field mesocosms}







% ---------------------------------------------------------------------------------------
% ---------------------------------------------------------------------------------------
% Results
% ---------------------------------------------------------------------------------------
% ---------------------------------------------------------------------------------------



\section*{Results}
 
The 1st, 2nd, and 3rd degree terms associated with midge larval density
were all statistically significant (Table \ref{tab:gpp}), 
indicating a nonlinear relationship between GPP and larval density.
This relationship was generally positive, 
although it saturated 
and was possibly negative at the highest densities (Figure \ref{fig:gpp}).
On day 7 of the experiment, 
all three sites had similar GPP-midge relationships and overall levels of GPP.
However, there were significant day $\times$ site, day $\times$ density,
and site $\times$ density interactions that manifested as a differences
between the sites on day 20.
In particular, the GPP-midge relationship was weaker on day 20 than on day 7 for sites
E2 and E5, while the midge effect at E3 remained largely similar.
The sites also diverged in overall GPP through time,
with E3 and E5 higher on day 20 than on day 7,
while E2 was lower.
These relationships corrected for the significantly positive effect 
of ambient temperature during the measurement incubations.
Therefore, the temporal patterns likely reflect real divergences 
between the productivity of the mesocosms at the three sites through time, 
rather than transient differences in ambient environmental conditions during the measurements.

The emergence rates of adults declined with initial larval density 
(Table \ref{tab:adult}), 
indicating negative density dependence.
Neither the main effect of site nor its interaction with density was significant.
Emergence rates increased with the GPP per initial larva 
(LRT: $\chi^2_{(1)}$=29.8; $\emph{P}$ < 0.001; Figure \ref{fig:adults}),
which is consistent with the hypothesis that negative density dependence is 
related in part to food limitation.
Accordingly, the positive effect of larval density on GPP reduced the strength of 
negative density dependence across the range of densities 
used in the experiment (Figure \ref{fig:feed}).
The positive effect of midges on their own emergence was maximized at intermediate
densities, 
which corresponds to where the effect of larval density on GPP was maximized 
(Figure \ref{fig:gpp}).
Above this density, 
the midge effect on GPP plateaued and perhaps even became slightly negative. 
Therefore, at the highest densities emergence rates converged on what they would be 
in the absence of midge effects on GPP. 
There was modest variation in the midge effect on density dependence among the three sites,
with the effect being greatest at E3.
This reflects the fact that the positive midge effect on GPP declined through time
at sites E2 and E5, while at E3 it remained largely consistent. 







% ---------------------------------------------------------------------------------------
% ---------------------------------------------------------------------------------------
% Discussion
% ---------------------------------------------------------------------------------------
% ---------------------------------------------------------------------------------------



\section*{Discussion}

Our field mesocosm experiment shows how ecosystem engineering can weaken negative 
density dependence for the population of engineers.
We found that \emph{T. gracilentus} larvae in  M\'{y}vatn have large positive 
effects on benthic primary production,
which previous studies have shown are driven at least in part by physical 
structure provided by the tubes in which the larvae reside 
\citep{hoelker2015, phillips2019}.
Because \emph{T. gracilentus} larvae feed on benthic diatoms \citep{ingvason2004},
simulation of benthic production increased the amount of production per individual,
which in turn weakened negative density-dependence arising from food limitation.


While the effect off midges on benthic production was generally positive in our experiment,
this effect was nonlinear and plateaued between 60,000 and 90,000 
individuals $\text{m}^{-2}$.
Consequently, 
the midge-mediated weakening of density dependence was maximized
at intermediate densities.
Given the suppression of algal biomass through grazing \citep{einarsson2016},
it is likely that the effect of midges on production becomes negative
at the highest densites observed in the lake (>200,000 $\text{m}^{-2}$).
This suggests that while midges weaken negative density dependence at moderate
densities, they may enhance it at the highest densities.
The nonlinearity of ecosytem engineer effects has previously been identified 
as an important factor in governing the effects of engineering on community dynamics 
\citep{bozec2013}.
Despite weakening densitity dependence at low to moderate densities,
midge engineering did not lead to positive densitiy dependence 
\citep[i.e., allee effects;][]{courchamp1999}.
This has important implications for their population dynamics,
as positive density dependence could lead to run-away or overcompensatory growth
\citep{turchin2003, cuddington2009}.

The midge effect on benthic producivity across the three sites 
was similar at the beginning of the experiment,
but diverged through time even after accounting for variation in ambient conditions
during the productivity measurements.
This suggests that there was a legacy of local conditions that affected the 
response of benthic producers to midge engineering.
While our experiment was not directly able to test for such legacies,
a plausible candidate is temperature,
which was consistently lowest at the site with the weakest response to midges.
Various studies have identified the role of environmental variation in mediating 
the strength and sign of ecosystem engineering \citep{wright2006,lathlean2017}.
Enduring legacies of environemtnal mediation are particularly important,
as they may serve to decouple the dynamics of the engineers 
and their community-wide effects
\citep{cuddington2011}.
Such decoupling may in tern alter the nature of density-dependent feedbacks in 
systems with ecosystem engineering \citep{cuddington2009}.




% ---------------------------------------------------------------------------------------
% ---------------------------------------------------------------------------------------
% Acknowledgments
% ---------------------------------------------------------------------------------------
% ---------------------------------------------------------------------------------------

\section*{Acknowledgments}

This work was supported by National Science Foundation grants 
DEB-1052160, DEB-1556208 to Anthony R. Ives,
and Graduate Research Fellowships DGE-1256259.
The M\'{y}vatn Research Station directed by \'{A}rni Einarsson
provided logistical and scientific support.
We thank Kristian Riley Book, Natalie Schmer, Bethany Smith, and Aspen Ward
for assistance with fieldwork.


% ---------------------------------------------------------------------------------------
% ---------------------------------------------------------------------------------------
% Literature Cited
% ---------------------------------------------------------------------------------------
% ---------------------------------------------------------------------------------------



\bibliographystyle{ecology.bst}
\clearpage

\bibliography{refs.bib}

\clearpage





% ---------------------------------------------------------------------------------------
% ---------------------------------------------------------------------------------------
% Tables & Figures
% ---------------------------------------------------------------------------------------
% ---------------------------------------------------------------------------------------

% ---------------------------------------------------------------------------------------
% Table I
% ---------------------------------------------------------------------------------------

\begin{table}
\caption{\label{tab:gpp}
LMM for mesocosm GPP. 
\emph{P}-values are from \emph{F}-tests with the Kenward-Roger correction.
The model included random effects for block ($\upsigma$ = 0.001)
and mesocosm identity ($\upsigma$ = 0.013), 
with residual standard deviation $\upsigma$ = 0.012.}
\setlength{\tabcolsep}{12pt}
\begin{tabular}{lllllll}
\toprule
& & \multicolumn{2}{c}{Type III} & & \multicolumn{2}{c}{Type II} \\
\cmidrule{3-4} \cmidrule{6-7}
Term & & \emph{F}_{(ndf,ddf)} & \emph{P} & & \emph{F}_{(ndf,ddf)} & \emph{P}\\
\midrule
temperature & & 31_{(1,41.2)} & <0.001 & & 31_{(1,41.2)} & <0.001\\

PAR & & 0.001_{(1,38.9)} & 0.981 & & 0.001_{(1,38.9)} & 0.981\\

period & & 0.35_{(1,41.3)} & 0.555 & & 79_{(1,41.3)} & <0.001\\

site & & 0.15_{(2,4.63)} & 0.861 & & 4_{(2,2.62)} & 0.161\\

density & & 12_{(1,37.8)} & 0.002 & & 13_{(1,37)} & <0.001\\

density^2 & & 5.6_{(1,37.4)} & 0.023 & &  & \\

density^3  & & 4.6_{(1,37.1)} & 0.039 & & & \\

period $\times$ site & & 10_{(2,41)} & <0.001 & & & \\

period $\times$ density & & 16_{(1,39.4)} & <0.001 & & & \\

site $\times$ density & & 4.4_{(2,36.6)} & 0.020 & & & \\
\bottomrule
\end{tabular}
\end{table}


\clearpage



% ---------------------------------------------------------------------------------------
% Table II
% ---------------------------------------------------------------------------------------

\begin{table}
\caption{\label{tab:adult}
GLMM for adult emergence rate, as a proportion of the initial number of individuals.
\emph{P}-values are from a bootstrapped likelihood-ratio test, 
calculated from 2000 simulated data sets.
The model included random effects for block ($\upsigma$ = 0.03)
and mesocosm identity ($\upsigma$ = 0.26).}
\setlength{\tabcolsep}{12pt}
\begin{tabular}{lllllll}
\toprule
& & \multicolumn{2}{c}{Type III} & & \multicolumn{2}{c}{Type II} \\
\cmidrule{3-4} \cmidrule{6-7}
Term & & \chi^2_{(df)} & \emph{P} & & \chi^2_{(df)} & \emph{P}\\
\midrule
density & & 0.0_{(1)} & 0.288 & & 31.9_{(1)} & <0.001\\

site & & 4.7_{(2)} & 0.217 & & 2.1_{(2)} & 0.422\\

density $\times$ site & & 4.5_{(2)} & 0.147 & & & \\
\bottomrule
\end{tabular}
\end{table}

\clearpage



% ---------------------------------------------------------------------------------------
% Figure 1
% ---------------------------------------------------------------------------------------

\begin{figure}
\centering
\includegraphics{../analysis/figures/p_sites.pdf}
\caption{\label{fig:sites}
Experimental sites. 
(a) The three sites covered a wide range of M\'{y}vatn's south basin.
Light gray areas indicate water, while white areas indicate land.
(b) Mean daily PAR and temperature were calculated by averaging half-hourly
measurements from two loggers deployed on the lake bottom for each site.
}
\end{figure}

\clearpage



% ---------------------------------------------------------------------------------------
% Figure 2
% ---------------------------------------------------------------------------------------

\begin{figure}
\centering
\includegraphics{../analysis/figures/p_gpp.pdf}
\caption{\label{fig:gpp}
GPP as a function of initial larval density.
The points show the observed data standardized to the mean water temperature
measured during the incubations.
The solid lines show fitted values from the third-degree polynomial LMM.
While the model provides a smooth fit to the data,
we connected the fitted values at the experimental densities with
straight lines to draw attention to discete levels at which the measurements were taken. 
The shaded regions show the standard errors estimated from the 
covariance matrix associated with the model fit.
}
\end{figure}

\clearpage



% ---------------------------------------------------------------------------------------
% Figure 3
% ---------------------------------------------------------------------------------------

\begin{figure}
\centering
\includegraphics{../analysis/figures/p_adults.pdf}
\caption{\label{fig:adults}
Adult emergence rate as a function of GPP per initial midge larva.
The points show the observed data, 
the line shows teh fitted values from a GLMM,
and the shaded regions show the standard errors estimated from the 
covariance matrix associated with the model fit.
}
\end{figure}

\clearpage

% ---------------------------------------------------------------------------------------
% Figure 4
% ---------------------------------------------------------------------------------------

\begin{figure}
\centering
\includegraphics{../analysis/figures/p_feed.pdf}
\caption{\label{fig:feed}
Consequences of midge effects on GPP for negative density-dependence.
The curves are derived from the conjunction of the LMM in Figure \ref{fig:gpp}
and the GLMM in Figure \ref{fig:adults}. 
The solid lines (``with midge effect'') show modeled emergence rates including the modeled
effect of midges on overall GPP,
while the dashed lines (``no midge effect'') exclude midge effects on overall GPP.
The difference between the two lines is a measure of the midge effect on density-dependent
emergence as mediated by midge effects on GPP. 
}
\end{figure}

\clearpage




% ---------------------------------------------------------------------------------------
% ---------------------------------------------------------------------------------------
% Supplement
% ---------------------------------------------------------------------------------------
% ---------------------------------------------------------------------------------------

% \section*{Supplement}

\end{document}
