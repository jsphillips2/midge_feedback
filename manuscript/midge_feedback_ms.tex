% ---------------------------------------------------------------------------------------
% ---------------------------------------------------------------------------------------
% Specifications
% ---------------------------------------------------------------------------------------
% ---------------------------------------------------------------------------------------

\documentclass[12pt]{article}
\usepackage[letterpaper, margin=1in]{geometry}
\usepackage{newtxtext,newtxmath}
\usepackage[math-style=ISO]{unicode-math}
\usepackage{fullpage}
\usepackage[authoryear,sectionbib]{natbib}
\linespread{1.7}
\usepackage[utf8]{inputenc}
\usepackage{lineno}
\usepackage{titlesec}
\titleformat{\section}[block]{\Large\bfseries\filcenter}{\thesection}{1em}{}
\titleformat{\subsection}[block]{\Large\itshape\filcenter}{\thesubsection}{1em}{}
\titleformat{\subsubsection}[block]{\large\itshape}{\thesubsubsection}{1em}{}
\titleformat{\paragraph}[runin]{\itshape}{\theparagraph}{1em}{}[. ]
\renewcommand{\refname}{Literature Cited}
\DeclareTextSymbolDefault{\dh}{T1}
\medmuskip=8mu 
\thickmuskip=10mu
\usepackage{graphicx}
\usepackage{booktabs}
\renewcommand{\thetable}{\Roman{table}}
\usepackage{caption}
\captionsetup{justification = raggedright,
              singlelinecheck = false,
              labelfont = bf}






% ---------------------------------------------------------------------------------------
% ---------------------------------------------------------------------------------------
% Title page
% ---------------------------------------------------------------------------------------
% ---------------------------------------------------------------------------------------

\title{Ecosystem engineering alters density-dependent feedbacks in an aquatic insect population}

\author{
Joseph S. Phillips$^{1,2,\dagger}$ \\
Amanda R. McCormick$^{1,3}$ \\
Jamieson C. Botsch$^{1}$ \\
Anthony R. Ives$^{1}$}

\usepackage{amsmath} % for split math environment

\date{}

\begin{document}

\raggedright
\setlength\parindent{0.25in}

\maketitle

\noindent{} 1. Department of Integrative Biology, University of Wisconsin, Madison, Wisconsin 53706 USA

\noindent{} 2. Department of Aquaculture and Fish Biology, H\'{o}lar University, Skagafj\"{o}r{\dh}ur 551 Iceland

\noindent{} 3. Department of Land, Air and Water Resources, 
University of California, Davis, CA, USA

\noindent{} $\dagger$ E-mail: joseph@holar.is

\bigskip

Running head: {Ecosystem engineering feedbacks}

\linenumbers{}

\clearpage





% ---------------------------------------------------------------------------------------
% ---------------------------------------------------------------------------------------
% Abstract
% ---------------------------------------------------------------------------------------
% ---------------------------------------------------------------------------------------

\section*{Abstract}

Ecosystem engineers have large impacts on the communities in which they live,
and these impacts may feed back to populations of engineers themselves.
In this study, we assessed the effect of ecosystem engineering 
on density-dependent feedbacks for midges in Lake M\'{y}vatn, Iceland. 
The midge larvae reside in the sediment and build silk tubes that provide 
a substrate for algal growth, thereby elevating benthic primary production.
Benthic algae are in turn the primary food source for the midge larvae,
setting the stage for the effects of engineering to feed back to the midges themselves.
Using a field mesocosm experiment manipulating larval midge densities,
we found a generally positive but nonlinear relationship between
density and benthic production.
Furthermore, 
adult emergence increased with the primary production per midge larva.
By combining these two relationships in a simple model,
we found that the positive effect of midges on benthic production 
weakened the negative density dependence at low to intermediate larval densities.
However, this benefit disappeared at high densities when midge consumption 
of primary producers exceeded their positive effects 
on primary production through ecosystem engineering.
Our results illustrate how ecosystem engineering can alter 
density-dependent feedbacks for engineer populations.

\bigskip

\textit{Keywords}: {benthic production; 
                    facilitation; 
                    feedbacks; 
                    interspecific interactions; 
                    macroinvertebrates;
                    \textit{T. gracilentus}}

\clearpage





% ---------------------------------------------------------------------------------------
% ---------------------------------------------------------------------------------------
% Introduction
% ---------------------------------------------------------------------------------------
% ---------------------------------------------------------------------------------------

\section*{Introduction}

Ecosystem engineering is a class of ecological interactions whereby 
one population affects others through alterations to the physical environment 
\citep{jones1994, wilby2002}.
Like all interspecific interactions, 
ecosystem engineering has the potential to generate feedbacks among various 
members of a community \citep{bertness1997,largaespada2012,donadi2014,sanders2014}.
For example, physical structure provided by coral can ameliorate competition with algae by
benefiting grazers that reduce algal abundance \citep{bozec2013}. 
As ecosystem engineers are, by definition, the source 
of engineering effects within an ecosystem, 
feedbacks between engineering and the engineers themselves are central 
to the dynamical consequences of engineering for the community as a whole 
\citep{hastings2007, sanders2014}. 

To understand the role of engineering feedbacks
for the population dynamics of ecosystem engineers,
it is useful to relate those feedbacks to the strength of density dependence
\citep{hastings2007, cuddington2009}.
Using a simple mathematical model,
\citealt{cuddington2009} showed that a wide range of dynamical behaviors
is possible for populations of ecosystem engineers,
including stable persistence, extinction, unbounded growth, and alternative states.
The dependence of engineering effects on population density and 
the subsequent feedback of engineering to density dependence 
are key factors determining the overall dynamical consequences of ecosystem engineering.
Despite their theoretical importance,
quantitative characterizations of density-dependent 
engineering feedbacks for natural populations are limited.
While previous studies have established that the engineering effects can scale 
with engineer density \citep{albertson2014}
and that engineering effects can feedback to engineer populations
\citep{largaespada2012, bozec2013, donadi2014},
the combination of these two elements is necessary for 
assessing the role of ecosystem engineering in 
mediating the density dependence of engineer population growth.
 
We assessed the effect of ecosystem engineering on the sign and magnitude
of density-dependent survival and emergence of the midge \textit{Tanytarsus gracilentus} 
(Diptera: Chironomidae) in Lake M\'{y}vatn, Iceland. 
The larvae of \textit{T. gracilentus} dwell in the sediment and build silk tubes that 
create a 3-dimensional substrate, 
similar to other benthic macroinvertebrates 
\citep{largaespada2012,donadi2014,hoelker2015}.
The midge tubes appear to ameliorate light limitation for epibenthic algae by 
providing additional surface area exposed to light,
which thereby elevates benthic primary production 
\citep{herren2017, phillips2019}.
Increased nutrient availability through excretion and bioturbation may also be relevant
\citep{hoelker2015}
although there is as yet no evidence of this for M\'{y}vatn.
The midge larvae feed on benthic algae (mainly diatoms) \citep{ingvason2004} and 
benefit from conditions conducive to algal growth \citep{wetzel2021}, 
which means that their enhancement of benthic production may benefit their own 
survival, emergence, and subsequent reproduction.
However, midge consumption may also reduce algal biomass,
potentially leading to intraspecific competition and negative density dependence 
\citep{einarsson2016}.
Indeed, the \textit{T. gracilentus} population 
in M\'{y}vatn shows large fluctuations in abundance 
that are likely driven by food limitation, although these fluctuations cannot be explained 
purely in terms of classical consumer-resource cycles \citep{ives2008}.
Characterizing the nature of density dependence is important for understanding 
the complex population dynamics of \textit{T. gracilentus},
making it a valuable case for exploring the effects of ecosystem engineering 
on density dependence of engineer populations.

To evaluate the role of ecosystem engineering 
on density-dependent survival and emergence in \textit{T. gracilentus},
we conducted a field mesocosm experiment across a range of experimental larval densities. 
This allowed us to directly quantify
(i) the relationship between benthic primary production and larval midge density and
(ii) the relationship between adult emergence rates and primary production per larval midge.
We then combined these two relationships with a simple model 
that allowed us to isolate the contribution of larval midge effects on 
primary production to their density-dependent survival and emergence.




% ---------------------------------------------------------------------------------------
% ---------------------------------------------------------------------------------------
% Methods
% ---------------------------------------------------------------------------------------
% ---------------------------------------------------------------------------------------


\section*{Methods}


M\'{y}vatn is a large ($37\text{km}^2$), shallow (mean depth: 2.5m), 
naturally eutrophic lake in northeastern Iceland (65°40’N 17°00’W) \citep{einarsson2004}.
It is separated into two ecologically distinct basins (north and south)
fed by ground water springs rich in N, P, and Si on the eastern side of the lake
\citep{einarsson2004}.
Our study was conducted in 2017 at three soft-substrate sites in the south basin  
(E2, E3, and E5 from west to east) 
that were selected to represent a range of ecological conditions 
and \textit{T. gracilentus} abundance (Figure S1a).
In sediment cores taken throughout the summer of 2017, E3 
had the highest Tanytarsini (including \textit{T. gracilentus}) densities 
(mean $\pm$ standard error: $69,058 \pm 14,595~\text{m}^{-2}$),
followed by E5 ($30,648 \pm 11,767$), and then E2 ($431 \pm 172$).
Maximum densities in M\'{y}vatn have exceeded $500,000~\text{m}^{-2}$ 
\citep{thorbergsdottir2004}.
In the summer of 2017, 
E2 was subject to an expanding epibenthic mat 
of filamentous green algae (Cladophorales) that was largely absent from E3 and E5.
Furthermore, E2 was consistently colder (mean difference 1$^{\circ}$C) 
than E3 and E5 during the experiment period (Figure S1b).
In contrast, photosynthetically active radiation (PAR) was similar among the sites,
due to their similar depths (E2: 2.8m; E3: 3.3m; E5: 2.6m)
and water clarities throughout the south basin in 2017.
Light and temperature data were collected with two loggers 
(HOBO Pendant, Onset Computer Corporation) deployed on the lake bottom at each site
and set to log every 30 minutes.
Light was measured as visual intensity
and approximately converted to PAR using a standard correction \citep{thimijan1983}.

We conducted our field mesocosm experiment using a design
similar to \cite{phillips2019}.
On 28 June 2017, we collected approximately 50 sediment cores from each 
of the three study sites using a Kajak corer. 
For each site, we pooled the sediment from the different cores while keeping the 
top 5cm (``top'') and next 10cm (``bottom'') separate.
We then sieved the sediment through either 125 (top) or 500$\mu \text{m}$ (bottom) mesh
to remove midge larvae; 
sieving also removed the surface Cladophorales abundant in cores from E2.
The sediment was left to settle for 4d in a cool and dark location.
We constructed the mesocosms by stocking the sediment into 
clear acrylic tubes (33cm height $\times$ 5cm diameter) 
sealed from the bottom with foam stoppers.
We first added 10cm of bottom sediment and then 5cm of top sediment, 
to mimic the layering in the lake.
The sediment layer of each mesocosm was wrapped with 4 layers of black plastic
to eliminate light from the sides.
Images of the mesocosms are shown in the Supplementary Materials (Figures S2 and S3).

On 3 July, 
we collected approximately 100 sediment cores at E3 and 
sieved them through 125$\mu \text{m}$ mesh
to collect Tanytarsini larvae; 
the vast majority were likely \textit{T. gracilentus},
although identification to the species level could not readily be done on live individuals.
The other Tanytarsini present in the lake are of the genus \textit{Micropsectra}; 
they are generally restricted to the largely profundal
southeastern region of the lake and unlikely 
to be represented in our experiment.
Tanytarsini progress through four instars before emerging as adults.
We attempted to select individuals the general size of second-instar larvae
to maximize the duration of the experiment before emergence.
On 4 July, we stocked the mesocosms with four densities of Tanytarsini larvae:
0, 50, 100, 200 per mesocosm (0, 25000, 51000, and 102000 $\text{m}^{-2}$). 
Each site $\times$ density combination had four replicates, for a total of 48 mesocosms.
We filled the mesocosms with water collected from near  M\'{y}vatn's southern shore
and gave the midges 24h to settle and rebuild their tubes
before deploying the mesocosms in the lake.
On 5 July, 
we distributed the mesocosms corresponding to each site onto two racks and then
deployed them at their respective sites on the lake bottom.
The tops of the mesocosms were left open to allow exchange between the mesocosms
and the lake water column.
While the open tops made it possible for midges to either leave or colonize,
the strong association between the number of midges stocked in the experiment and 
the number recovered at the end (see below) mitigates this concern.
Furthermore, observations from laboratory experiments \citep[e.g.,][]{wetzel2021}
suggest that \textit{T. gracilentus} larvae are largely sedentary 
once they have built tubes.

On 10 and 11 July (days 5 and 6), 
we estimated gross primary production (GPP) in the mesocosms by measuring the change in
dissolved oxygen (DO) concentration during sealed incubations 
\citep[similar to][]{phillips2019}.
The incubations were conducted in situ at the respective sites to incorporate
spatial variation in ambient conditions, such as light and temperature.
Each mesocosm was first incubated under ambient light to give an
estimate of net ecosystem production (NEP),
followed by a dark incubation with the top of each mesocosm wrapped
in 4 layers of black plastic to give an estimate of ecosystem respiration (ER).
NEP + ER gives an estimate of GPP, assuming that ER is the same during both the light
and dark incubations. Half of the mesocosms at each site were incubated on 10 July,
while the other half were incubated on 11 July;
all of the mesocosms incubated on a given day for a given site were on the same 
experimental rack and so constituted a block.
The incubations lasted 3--5h, 
and the tops of the mesocosms were sealed with rubber stoppers for the duration.
DO was measured using a handheld probe (ProODO, YSI, Yellow Springs, Ohio, USA),
and we gently stirred the water within each mesocosm to homogenize it 
before taking the reading. 
We repeated the incubation procedure on 21 and 23 July (days 16 and 18).
Due to difficult weather, 
we were unable to perform the incubations at the respective sites.
Therefore, on 21 July all of the mesocosms were moved to a bay on the
southern shore of the south basin (depth $\approx$ 1.7m; Figure S1).
The light incubations lasted 3--5h,
while the dark incubations lasted 4--10h.
While variation in incubation duration was not ideal,
the DO in the dark incubations remained above anoxic conditions 
(minimum DO >10 $\text{mg L}^{-1}$). 
We converted GPP to units of $\text{mg}~\text{O}_2~\text{m}^{-2}~\text{h}^{-1}$,
accounting for incubation duration and water column depth within each mesocosm.

On 23 July, shortly before the expected time of midge emergence,
we removed the mesocosms from the lake and covered the top of each with mesh
to catch adult midges as they emerged. 
We kept the mesocosms outdoors in baths of cold tap water 
to moderate temperature.
Every 1--3d for the next 13d, we collected the emerging adults from the mesocosms.
While these were not individually identified, the vast majority appeared to be Tanytarsini.
Furthermore, there was a strong association between the number of Tanytarsini larvae 
stocked in the mesocosms and the number of adults that emerged 
(Spearman rank correlation of 0.82; $\textit{P}$ < 0.0001).

We quantified the relationship between GPP and larval density using a 
linear mixed model (LMM). 
The model included initial density (numeric), site (three levels),
incubation day (two levels; either days 5--6 or 16--18), 
and their two-way interactions as fixed effects.
Because we expected the relationship between GPP and initial density to be nonlinear,
we also included 2nd and 3rd order polynomial terms for initial density 
(without any interactions) in the model.
We chose a third-degree polynomial because this gave the same number of parameters 
to estimate as would have been the case if each of the four density levels were treated 
categorically (including the intercept). 
The polynomial regression allowed us to treat density as a numeric variable 
and simplify the model by only including interactions with the linear density term.
We accounted for variation in ambient conditions during the incubations by including
linear terms for PAR and temperature estimated for each block at each site.
Finally, we included random effects for experimental rack and mesocosm identity to account
for blocking and repeated measures.

We used a binomial generalized linear mixed model (GLMM) to analyze variation 
in the number of midges that emerged as adults
out of a given number of initial larvae
(excluding the zero treatment). 
We included initial density (numeric), site (three levels), and their interaction
as fixed effects. 
We included random effects for experimental rack and mesocosm identity to account
for blocking and potential overdispersion, respectively;
the latter was equivalent to assuming the residuals followed a 
logit-normal binomial distribution.

Our goal in this study was to assess the effect of midge ecosystem engineering 
on their emergence.  
To visualize this, we fit a logit-normal binomial GLMM 
to the number of midges that emerged as adults
out of a given number of initial larvae,
with GPP per initial larva (averaged across the two sample dates for each mesocosm)
as the sole fixed effect. 
We then projected the number of emerging adults under two scenarios: 
(i) using predicted values of GPP as a function of site
and larval density treatment according to the polynomial LMM described above, 
and (ii) using predicted values of GPP as a function of site, 
but with larval density set to zero (on the natural scale). 
Scenario (ii) implies that GPP per larva declines across the midge treatments 
purely due to the greater number of larvae over which production is distributed. 
Scenario (i) differs from scenario (ii) by also including direct effects of 
larval density on GPP. 
The difference between scenarios (i) and (ii) gives a measure of the effect 
of larval density on the density dependence of survival and emergence 
with and without the effects of midge larvae on GPP

Statistical analyses were conducted in R 4.0.3 \citep{r2020},
using the \texttt{lme4} package to fit the LMM and GLMMs \citep{lme4}.
We calculated $\textit{P}$-values with $\textit{F}$-tests using
the Kenward-Roger correction for the LMM \citep{halekoh2014kenward} 
using the \texttt{car} package \citep {fox2019}
and parametric-bootstrapped likelihood-ratio tests (LRTs) based on 2000 simulations
for the GLMMs (\texttt{simulate} function in the native \texttt{stats} package).
We report both Type III and Type II tests unless otherwise noted,
to balance concerns of inflated Type I errors that can occur when dropping 
terms with the poor statistical inference that can come from overparameterized models
\citep{zuur2009}.
Model formulas for statistical analyses are given in the Supplementary Materials
(equations S1-S3).





% ---------------------------------------------------------------------------------------
% ---------------------------------------------------------------------------------------
% Results
% ---------------------------------------------------------------------------------------
% ---------------------------------------------------------------------------------------

\section*{Results}

The mesocosm experiment revealed a nonlinear relationship between GPP and larval density,
as the first, second, and third-degree terms associated with midge larval density 
were all statistically significant (Table \ref{tab:gpp}). 
This relationship was generally positive, 
although it saturated 
and was possibly negative at the highest densities (Figure \ref{fig:comb}a).
On days 5--6 of the experiment, 
all three sites had similar GPP-density relationships and overall levels of GPP.
However, there were statistically significant day $\times$ site, day $\times$ density,
and site $\times$ density interactions that manifested as differences
between the sites on days 16--18.
For sites E2 and E5, the GPP-density relationship was weaker 
on days 16--18 than on days 5--6, 
while the density effect at E3 remained largely similar through time.
Furthermore, GPP for E3 and E5 was higher on days 16--18 than on days 5--6, 
while for E2 it was lower.
The observed data and model fits were standardized to mean ambient temperature 
during the incubations by using the statistically significant temperature coefficient 
estimated from the LMM. 
Therefore, the temporal patterns likely reflect real divergences 
between the productivity of the mesocosms at the three sites through time, 
rather than transient differences in ambient environmental conditions 
during the measurements.

The proportion of individuals that emerged as adults declined with initial larval density 
(Type II LRT: $\chi^2_{(1)}$=31.9; $\textit{P}$ < 0.001), 
indicating negative density dependence.
Neither the main effect of site
($\chi^2_{(2)}$=2.1; $\textit{P}$ = 0.422)
nor its interaction with density 
($\chi^2_{(2)}$=4.5; $\textit{P}$ = 0.147) were statistically significant.
Proportional emergence increased with the GPP per initial larva 
(LRT: $\chi^2_{(2)}$=29.8; $\textit{P}$ < 0.001; Figure \ref{fig:comb}b),
which is consistent with the hypothesis that negative density dependence is 
related in part to food limitation.

We assessed the effect of larval density on adult emergence 
by projecting the number of emerging midges under two scenarios: 
(i) including the effect of larval density on overall GPP
and (ii) assuming that GPP was constant and therefore was not affected by midge larvae.  
Differences between the two scenarios quantified the consequences of midge effects on GPP
for density-dependent emergence.
The positive effect of larvae on GPP
reduced the negative effect of larval density
on the proportion of individuals that emerged as adults
(Figure \ref{fig:feed}a).
However, because the midge effect on GPP plateaued at high initial midge density 
(Figure \ref{fig:comb}a),
proportional emergence at high density converged on what it would be 
without the midge effect on GPP.
There was modest variation in the midge effect 
on density dependence among the three sites, with the effect being greatest at E3. 
This reflects the fact that the positive midge effect 
on GPP declined through time at E2 and E5, 
while at E3 it remained relatively constant. 
The nonlinear effect of midge larvae on GPP resulted in a hump-shaped relationship
between total emergence and larval density (Figure \ref{fig:feed}b),
indicating that the greatest emergence occurred at intermediate densities.





% ---------------------------------------------------------------------------------------
% ---------------------------------------------------------------------------------------
% Discussion
% ---------------------------------------------------------------------------------------
% ---------------------------------------------------------------------------------------


\section*{Discussion}

Our field mesocosm experiment showed that an ecosystem engineer population 
has positive effects on its food resources, 
thereby establishing a positive feedback 
that alters density-dependent survival and emergence. 
We found that Tanytarsini larvae in  M\'{y}vatn have large positive 
effects on benthic primary production,
which previous studies have shown are driven at least in part by physical 
structure provided by the silk tubes in which the larvae reside 
\citep{hoelker2015, phillips2019}.
Because midge larvae feed on benthic diatoms \citep{ingvason2004},
stimulation of benthic production increased the amount of food per individual midge,
which in turn weakened negative density-dependence arising from food limitation
at moderate midge densities.
However, the amelioration of negative density dependence largely disappeared at high densities,
presumably due to midge suppression of diatom growth through consumption.

The nonlinear feedback of midge engineering on primary production meant 
that the weakening of density dependence in emergence was greatest at intermediate densities. 
Given the suppression of algal biomass through grazing \citep{einarsson2016},
it is likely that the effect of midges on production becomes negative
at the highest densities observed in the lake.
This suggests that while the feedback through midge engineering weakens 
negative density dependence at moderate densities, 
this is not enough to overcome the consumptive effect of midges at high densities. 
The nonlinearity of ecosystem engineer effects has previously been identified 
as an important factor in governing the effects of engineering on community dynamics 
\citep{bozec2013}.
Despite weakening density dependence at low to moderate densities,
midge engineering did not lead to positive density dependence in per capita emergence.
However, the nonlinear engineering feedback did result in a hump-shaped relationship
between total emergence and larval density, 
which could lead to overcompensatory dynamics and accentuated fluctuations as observed
for the \textit{T. gracilentus} population
\citep{turchin2003, cuddington2009}.
It is important to note that our study does not include information on reproduction
and so does not fully capture the effect of density  dependence on the full life cycle.
Nonetheless, given the short duration of the adult stage and high egg production per adult 
\textit{T. gracilentus}, 
it is plausible that the dynamics of the ``engineering stage'' (i.e., the larva)
are principally driven by larval survival, of which adult emergence is a direct extension.
While ecosystem-engineering effects of other benthic invertebrates may differ from 
tube-building midges by being concentrated in the adult stage 
\citep[e.g., mussels;][]{largaespada2012},
their population dynamics may be similarly dominated by the
engineering stage (i.e., adults).

The midge effect on benthic productivity across the three sites 
was similar at the beginning of the experiment
but diverged through time even after accounting for variation in ambient conditions
during the productivity measurements.
Furthermore, the engineering feedback subtly differed between sites,
with midge emergence experiencing the greatest benefit from engineering at E3,
which was the site with the greatest ambient densities at the beginning of the experiment.
This suggests that local environmental context altered the consequences 
of ecosystem engineering, 
and these alterations had the potential to persist through time.
While our experiment was not directly able to test for such legacies,
plausible candidate causes are temperature (which was consistently lower at E2),
and sediment nutrient concentrations that may vary across locations 
due to proximity to the spring inputs on the eastern side of the lake.
These environmental variables could have altered algal 
community composition or abundance
\citep{mccormick2019},
thereby mediating their capacity to respond to the 
amelioration of light limitation provided 
by the expanded surface area of midge tubes
\citep{phillips2019}.
Environmental mediation of the sign and magnitude of ecosystem engineering 
has been well established in other settings 
\citep{wright2006,lathlean2017}
and may serve to decouple the dynamics of the engineers and their community-wide effects.
Therefore, further work characterizing how density-dependent feedbacks vary across 
environmental contexts is essential for understanding the dynamics of ecosystem engineering
\citep{cuddington2009}.





% ---------------------------------------------------------------------------------------
% ---------------------------------------------------------------------------------------
% Acknowledgments
% ---------------------------------------------------------------------------------------
% ---------------------------------------------------------------------------------------

\section*{Acknowledgments}

This work was supported by National Science Foundation grants 
DEB-1052160, DEB-1556208 to ARI,
and Graduate Research Fellowships DGE-1256259 and DGE-1747503.
We thank the M\'{y}vatn Research Station directed by \'{A}. Einarsson
for general research support and K.R. Book, M. McCary, N. Schmer, B. Smith, and A. Ward
for assistance with fieldwork.


% ---------------------------------------------------------------------------------------
% ---------------------------------------------------------------------------------------
% Literature Cited
% ---------------------------------------------------------------------------------------
% ---------------------------------------------------------------------------------------


\bibliographystyle{ecology.bst}

\bibliography{refs.bib}

\clearpage

% ---------------------------------------------------------------------------------------
% ---------------------------------------------------------------------------------------
% Tables & Figures
% ---------------------------------------------------------------------------------------
% ---------------------------------------------------------------------------------------

% ---------------------------------------------------------------------------------------
% Table I
% ---------------------------------------------------------------------------------------

\begin{table}
\caption{\label{tab:gpp}
LMM for mesocosm GPP. 
\textit{P}-values are from \textit{F}-tests with the Kenward-Roger correction.
The model included random effects for block ($\upsigma$ = 0.001)
and mesocosm identity ($\upsigma$ = 0.013), 
with residual standard deviation $\upsigma$ = 0.012.
The factor ``day'' had two levels (either day 5--6 or 16--18).}
\setlength{\tabcolsep}{12pt}
\begin{tabular}{lllllll}
\toprule
& & \multicolumn{2}{c}{Type III} & & \multicolumn{2}{c}{Type II} \\
\cmidrule{3-4} \cmidrule{6-7}
Term & & \textit{F}_{(ndf,ddf)} & \textit{P} & & \textit{F}_{(ndf,ddf)} & \textit{P}\\
\midrule
temperature & & 31_{(1,41.2)} & <0.001 & & 31_{(1,41.2)} & <0.001\\

PAR & & 0.001_{(1,38.9)} & 0.981 & & 0.001_{(1,38.9)} & 0.981\\

day & & 0.35_{(1,41.3)} & 0.555 & & 79_{(1,41.3)} & <0.001\\

site & & 0.15_{(2,4.63)} & 0.861 & & 4_{(2,2.62)} & 0.161\\

density & & 12_{(1,37.8)} & 0.002 & & 13_{(1,37)} & <0.001\\

density^2 & & 5.6_{(1,37.4)} & 0.023 & &  & \\

density^3  & & 4.6_{(1,37.1)} & 0.039 & & & \\

day $\times$ site & & 10_{(2,41)} & <0.001 & & & \\

day $\times$ density & & 16_{(1,39.4)} & <0.001 & & & \\

site $\times$ density & & 4.4_{(2,36.6)} & 0.020 & & & \\
\bottomrule
\end{tabular}
\end{table}


\clearpage


% ---------------------------------------------------------------------------------------
% Captions
% ---------------------------------------------------------------------------------------

\textbf{Figure 1.}
(a) GPP as a function of initial larval density.
The points show the observed data standardized to the mean water temperature 
using the temperature coefficient from the LMM.
The solid lines show fitted values from the third-degree polynomial LMM.
While the model provides a smooth fit to the data,
we connected the fitted values at the experimental densities with
straight lines to draw attention to discrete levels at which the measurements were taken. 
The shaded regions show the standard errors estimated from the 
covariance matrix associated with the model fit.
(b) Adult emergence rate as a function of GPP ($\text{mg O}_2~\text{m}^{-2}~\text{h}^{-1}$) 
per initial midge larva, 
averaged between the two time periods for each mesocosm 
and standardized to the mean water temperature 
using the temperature coefficient from the LMM.
The points show the observed data, 
the line shows the fitted values from a GLMM,
and the shaded region shows the standard errors estimated from the 
covariance matrix associated with the model fit.


\textbf{Figure 2.}
Consequences of midge effects on GPP for negative density dependence of 
either (a) the proportion of larvae that emerged as adults or 
(b) the total number that emerged.
The curves are derived from the conjunction of the LMM in Figure \ref{fig:comb}a
and the GLMM in Figure \ref{fig:comb}b. 
The solid lines (``with midge effect'') show modeled emergence rates including the modeled
effect of midges on overall GPP,
while the dashed lines (``no midge effect'') exclude midge effects on overall GPP.
The difference between the two lines is a measure of the midge effect on density-dependent
emergence as mediated by midge effects on GPP. 


% ---------------------------------------------------------------------------------------
% GPP + Emergence
% ---------------------------------------------------------------------------------------

\begin{figure}
\centering
\linespread{1}
\includegraphics{p_comb.pdf}
\caption{\label{fig:comb}
% (a) GPP as a function of initial larval density.
% The points show the observed data standardized to the mean water temperature
% measured during the incubations.
% The solid lines show fitted values from the third-degree polynomial LMM.
% While the model provides a smooth fit to the data,
% we connected the fitted values at the experimental densities with
% straight lines to draw attention to discrete levels at which the measurements were taken. 
% The shaded regions show the standard errors estimated from the 
% covariance matrix associated with the model fit.
% (b) Adult emergence rate as a function of GPP ($\text{mg O}_2~\text{m}^{-2}~\text{h}^{-1}$) 
% per initial midge larva.
% The points show the observed data, 
% the line shows the fitted values from a GLMM,
% and the shaded region shows the standard errors estimated from the 
% covariance matrix associated with the model fit.
}
\end{figure}

\clearpage




% ---------------------------------------------------------------------------------------
% Feedback
% ---------------------------------------------------------------------------------------

\begin{figure}
\centering
\linespread{1}
\includegraphics{p_feed.pdf}
\caption{\label{fig:feed}
% Consequences of midge effects on GPP for negative density dependence of 
% either (a) the proportion of larvae that emerged as adults or 
% (b) the total number that emerged.
% The curves are derived from the conjunction of the LMM in Figure \ref{fig:comb}a
% and the GLMM in Figure \ref{fig:comb}b. 
% The solid lines (``with midge effect'') show modeled emergence rates including the modeled
% effect of midges on overall GPP,
% while the dashed lines (``no midge effect'') exclude midge effects on overall GPP.
% The difference between the two lines is a measure of the midge effect on density-dependent
% emergence as mediated by midge effects on GPP. 
}
\end{figure}

\clearpage

\end{document}
