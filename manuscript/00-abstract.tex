
\section*{Abstract}

Ecosystem engineers have large impacts on the communities in which they live,
and these impacts may feed back to populations of engineers themselves.
In this study, we assessed the effect of ecosystem engineering 
on density-dependent feedbacks for midges in Lake M\'{y}vatn, Iceland. 
The midge larvae reside in the sediment and build silk tubes that provide 
a substrate for algal growth, thereby elevating benthic primary production.
Benthic algae are in turn the primary food source for the midge larvae,
setting the stage for the effects of engineering to feed back to the midges themselves.
Using a field mesocosm experiment manipulating larval midge densities,
we found a generally positive but nonlinear relationship between
density and benthic production.
Furthermore, 
adult emergence increased with the primary production per midge larva.
By combining these two relationships in a simple model,
we found that the positive effect of midges on benthic production 
weakened the negative density dependence at low to intermediate larval densities.
However, this benefit disappeared at high densities when midge consumption 
of primary producers exceeded their positive effects 
on primary production through ecosystem engineering.
Our results illustrate how ecosystem engineering can alter 
density-dependent feedbacks for engineer populations.


