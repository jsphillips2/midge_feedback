
\section*{Abstract}

In order to undertand the effects of ecosystem engineering on community dynamics,
it is important to characterize density-dependence feedbacks 
to populations of engineers themselves.
In this study, we assessed the sign and magnitude of ecosystem engineering effects 
on density dependence in 
midges in the naturally eutrophic Lake M\'{y}vatn. 
The midge larvae reside in the sediment and build silk tubes that provide 
a substrate for algal growth, thereby elevating benthic primary production.
Benthic algae are in turn the primary food source for the midge larvae,
setting the stage for the effects of engeering to feed back to thte midges themselves.
Using a field mesocosm experiment with a range of larval midge densities,
we found a generally positive but highly nonlinear relationship between
density and benthic production.
Furthermore, 
adult emergence increased with the primary production per initial midge larvae.
By combining these two relationships in a simple model,
we found that the positive effect of midges weakened the strength of negative density
dependence at low to intermediate larval densities.
However, this benefit disappeared at high densities due to the the nonlinearity of 
midge effects on production.
Our results illustrate how ecosystem engineering can alter the strength 
of density dependence for engineer populations.


