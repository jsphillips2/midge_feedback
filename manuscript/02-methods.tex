

\section*{Methods}


M\'{y}vatn is a large ($37\text{m}^2$), shallow (mean depth: 2.5m), 
naturally eutrophic lake in northeastern Iceland (65°40’N 17°00’W) \citep{einarsson2004}.
It is separated into two, ecologically distinct basins (north and south).
Our study was conducted in 2017 at three sites (E2, E3, and E5) in the south basin 
that were selected to represent a range of ecological conditions (Figure \ref{fig:sites}).
These sites all have soft substrate that is suitable for \emph{T. gracilentus},
although they can differ in larval densities. 
In sediment cores taken throughout the summer of 2017, E3 had the highest densities 
(mean $\pm$ standard error: $69,058 \pm 14,595~\text{m}^{-2}$),
followed by E5 ($30,648 \pm 11,767$), and then E2 ($431 \pm 172$).
Maximum densities in M\'{y}vatn have exceeded $200,000~\text{m}^{-2}$ \citep{lindegaard1979}.
In the summer of 2017, 
E2 was subject an expanding mat of filamentous green algae (Cladophorales) 
that was largely absent from E3 and E5 (J. Phillips; personal observation).
Furthermore, E2 was substantially colder during the experiment period than 
E3 and E5 (Figure \ref{fig:sites}).
In contrast, photosynthetically active radiation (PAR) was similar between the sites,
due to their similar depths (E2: 2.8m; E3: 3.3m; E5: 2.6m)
and similar water clarity throughout the south basin in 2017.
Light and temperature data were collected with two loggers 
(HOBO Pendant, Onset Computer Corporation) deployed on the lake bottom at each site
and set to log every 30 minutes.
PAR was recorded as visual light intensity (lux)
and approximately converted to PAR using a standard scaling factor \citep{thimijan1983}.

We conducted our field mesocosm experiment using a design
similar to \citep{phillips2019}.
On 28 June 2017, we collected sediment cores from the three study sites using a Kajak corer. 
For each site, we pooled the sediment from the different cores while keeping the 
top 5cm (``top'') and next 10cm (``bottom'') separate.
We then sieved the sediment through either 125 (top) or 500$\mu \text{m}$ (bottom) mesh
to remove midge larva.
The sediment was left to settle for 4 days in a cool, dark, location.
We constructed the mesocosms by stocking the sediment into 
clear acrylic tubes (33cm height $\times$ 5cm diameter) 
sealed from the bottom with foam stoppers.
We fist added 10cm of bottom sediment and then 5cm of top sediment, 
to mimic the layering in the lake.
The sediment layer of each mesocosm was wrapped with 4 layers of black plastic
to eliminate light from the sides of the mesocosms.

On 3 July, 
we took sediment cores at E3 and sieved them through 125 $\mu \text{m}$
to collect Tanytarsini larvae (the vast majority of which were likely \emph{T. gracilentus}).
Tanytarsini progress through four instars before emerging as adults.
We attempted to select individuals the general size of second instar larvae
to maximize the duration of the experiment before emergence.
The following day we stocked the mesocosms with four densities of Tanytarsini larvae:
0, 50, 100, 200 per mescosm (0, 25000, 51000, and 102000 $\text{m}^{-2}$). 
We then filled the mesocosms with water collected from the southern shore of M\'{y}vatn's
south basin and gave the midges 24 h to settle before deploying in the lake.
On 5 July, 
we distributed the mesocosms corresponding to each site onto two racks and then
deployed them at their respective sites on the lake bottom.
The tops of the mesocosms were left open to allow exchange between the mesocosms
and the lake water column.

On 10 and 11 July, 
we estimated gross primary production (GPP) in the mesocosms by measurnig the change in
dissolved oxygen (DO) concentration during sealed incubations \citep{hall2017}.
The incubations were conducted in situ at the respective sites to incorporate
spatial variation in ambient conditions, such as light and temperature.
Each mesocosm was first incubated under ambient light to give an
estimate of net ecosystem production (NEP),
followed by an incubation under dark conditions produced by wrapping each mesocosm 
in 4 layers of black plastic to give an estimate of ecosystem respiration (ER).
NEP + ER gives an estimate of GPP, assuming that ER is the same during both the light
and dark incubations. Half of the mesocosms at each site were incubated on 10 July,
while the other half were incubated on 11 July;
all of the mesocosms incubated on a given day for a given site were on the same 
experimental rack and so constituted a ``block''.
The incubations lasted between approximately 3 and 5 hours, 
and the tops of the mesocosms were sealed with rubber stoppers for the duration.
DO was measured using a handheld probe (ProODO, YSI, Yellow Springs, Ohio, USA),
and we gentled stirred the water within each mesocosm to homogenize it 
before taking the reading. 
We repeated the incubation procedure on 21 and 23 July.
For logistical reasons, we were unable to perform the incubations at the respective sites.
Therefore, on 21 July all of the mesocosms were moved to a common location on the
southern shore of the south basin (depth $\approx$ 1.7m).
The incubations lasted between approximately 3 and 5 hours,
while the dark incubations lasted between 4 and 10 hours.
While variation in incubation duration was not ideal,
the amount of DO in the dark incubations remained above anoxic conditions 
(minimum DO >10 $\text{mg}^{-1}$). 
We converted GPP to units of $\text{mg}~\text{O}_2~\text{m}^{-2}~\text{h}^{-1}$,
accounting for incubation duration and water column depth within each mesocosm.

On 23 July, shortly prior to when we expected midges to begin emerging from the mesocosms,
we removed the mesocosms from the lake and secured a mesh cover to the top of each
to catch adult midges as they emerged. 
We kept the mesocosms outdoors in mesh tents,
using water baths of cold tap water to moderate the temperature of the mesocosms.
The water baths had a depth of approximately 18cm,
which was sufficient to cover the sediment portion of each mesocosm while leaving the
tops exposed to the air to allow emergence.
Every 1-3 days for the next 13 days, we collected the emerging adults from the mesocosms.
While these were not individually identified, the vast majority appeared to be Tanytarsini.
Furthermore, there was a strong association between the number of Tanytarsini larvae 
stocked in the mesocosms and the number of adults that emerged 
(Spearman rank correlation of 0.82; $\emph{P}$ < 0.0001; including the zero density treatment).

We quantified the relationship between GPP and initial larval density using a 
linear mixed model (LMM). 
The model included initial density (four levels), site (three levels),
incubation period (two levels; either 10-11 or 21-23 July), 
and their two-way interactions as fixed effects.
Because we expected the relationship between GPP and initial density to be nonlinear,
we also included 2nd and 3rd order polynomial terms for initial density 
(without any interactions) in the model.
We choose a third degree polynomial because this gave the same number of parameters 
to estimate as would have been the case if each of the four treatments were treated as discrete
levels (including the intercept). 
The polynomial regression had the advantages of 
(a) allowing us to treat density as a numeric
variable and 
(b) of allowing us to simplify the model by only allowing interactions with the linear
density term.
We accounted for variation in ambient conditions during the incubations by including
linear terms for PAR and temperature estimated for each block at each site.
Finally, we included random effects for experimental rack and mesocosm identity to account
for blocking and repeated measures, respectively.

We used a binomial generalized linear mixed model (GLMM) to analyze variation 
in the proportion of the initial number of midge larvae that emerged as adults from each mesocosm 
(excluding the zero treatment). 
We included initial density (three levels), site (three levels), and their interaction
as fixed effects. 
We included random effects for experimental rack and mesocosm identity to account
for blocking and potential overdispersion, respectively;
the latter was equivalent to assuming the residuals followed a logit-normal-binomial distribution.
To assess the potential for larval midge effects on GPP to alter the strength of density-dependent
emergence, 
we fit a GLMM similar to the one described above, 
but with GPP per initial midge larva as the sole fixed effect. 
We then generated ``predicted'' values of adult emergence rates under two scenarios 
(1) using predicted values of GPP as a function of site and density treatment according
to the polynomial LMM described above, 
and (2) using predicted values of GPP fixed across larval density, but including variation 
across sites, based on the polynomial LMM evaluated at the mean larval density.
In scenario (2), GPP per larva declined across the midge treatments purely due to 
the partitioning of GPP across a greater number of individuals.
Scenario (1) included this decline in GPP due to partitioning, but also included
any positive or negative effects of larval density on GPP itself.
The difference between scenarios (1) and (2) gave a measure of the effect of larval
density on the strength of density dependence as mediated through midge effects on GPP.

Statistical analyses were conducted in R 4.0.0,
using the ``lme4'' package to fit the LMM and GLMMs.
We calculated $\emph{P}$-values with $\emph{F}$-tests using
the Kenward-Roger correction for the LMM (``Anova'' function in the``car'' package)
and with parametric-bootstrapped likelihood-ratio tests (LRTs) based on 2000 simulations
for the GLMMs (``simulate'' function in the native ``stats'' package).
We used both Type III and Type II tests,
to balance concerns with inflated Type-I errors that can occur when dropping 
terms with the poor statistical inference than can come from overparameterized models.









