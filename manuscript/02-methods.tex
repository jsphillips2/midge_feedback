

\section*{Methods}


M\'{y}vatn is a large ($37\text{m}^2$), shallow (mean depth: 2.5m), 
naturally eutrophic lake in northeastern Iceland (65°40’N 17°00’W) \citep{einarsson2004}.
It is separated into two ecologically distinct basins (north and south).
Our study was conducted in 2017 at three south basin sites (E2, E3, and E5) 
with soft substrate that were selected to represent a range of ecological conditions 
and \emph{T. gracilentus} abundance (Figure \ref{fig:sites}).
In sediment cores taken throughout the summer of 2017, E3 
had the highest Tanytarsini (including \emph{T. gracilentus}) densities 
(mean $\pm$ standard error: $69,058 \pm 14,595~\text{m}^{-2}$),
followed by E5 ($30,648 \pm 11,767$), and then E2 ($431 \pm 172$).
Maximum densities in M\'{y}vatn have exceeded $500,000~\text{m}^{-2}$ 
\citep{thorbergsdottir2004}.
In the summer of 2017, 
E2 was subject to an expanding mat of filamentous green algae (Cladophorales) 
that was largely absent from E3 and E5.
Furthermore, E2 was consistently colder (mean difference 1$^{\circ}$C) 
than E3 and E5 during the experiment period (Figure \ref{fig:sites}).
In contrast, photosynthetically active radiation (PAR) was similar among the sites,
due to their similar depths (E2: 2.8m; E3: 3.3m; E5: 2.6m)
and water clarities throughout the south basin in 2017.
Light and temperature data were collected with two loggers 
(HOBO Pendant, Onset Computer Corporation) deployed on the lake bottom at each site
and set to log every 30 minutes.
Light was measured as visual intensity
and approximately converted to PAR using a standard correction \citep{thimijan1983}.

We conducted our field mesocosm experiment using a design
similar to \cite{phillips2019}.
On 28 June 2017, we collected sediment cores from the three study sites using a Kajak corer. 
For each site, we pooled the sediment from the different cores while keeping the 
top 5cm (``top'') and next 10cm (``bottom'') separate.
We then sieved the sediment through either 125 (top) or 500$\mu \text{m}$ (bottom) mesh
to remove midge larva; 
sieving also removed the surface Cladophorales abundant in cores from E2.
The sediment was left to settle for 4d in a cool and dark location.
We constructed the mesocosms by stocking the sediment into 
clear acrylic tubes (33cm height $\times$ 5cm diameter) 
sealed from the bottom with foam stoppers.
We first added 10cm of bottom sediment and then 5cm of top sediment, 
to mimic the layering in the lake.
The sediment layer of each mesocosm was wrapped with 4 layers of black plastic
to eliminate light from the sides.

On 3 July, 
we took sediment cores at E3 and sieved them through 125$\mu \text{m}$ mesh
to collect Tanytarsini larvae; the vast majority were likely \emph{T. gracilentus},
although identification to the species level could not readily be done on live individuals.
Tanytarsini progress through four instars before emerging as adults.
We attempted to select individuals the general size of second instar larvae
to maximize the duration of the experiment before emergence.
On 4 July, we stocked the mesocosms with four densities of Tanytarsini larvae:
0, 50, 100, 200 per mesocosm (0, 25000, 51000, and 102000 $\text{m}^{-2}$). 
Each site $\times$ density combination had four replicates, for a total of 48 mesocosms.
We filled the mesocosms with water collected from near  M\'{y}vatn's southern shore
and gave the midges 24h to settle before deploying the mesocosms in the lake.
On 5 July, 
we distributed the mesocosms corresponding to each site onto two racks and then
deployed them at their respective sites on the lake bottom.
The tops of the mesocosms were left open to allow exchange between the mesocosms
and the lake water column.

On 10 and 11 July (days 5 and 6), 
we estimated gross primary production (GPP) in the mesocosms by measuring the change in
dissolved oxygen (DO) concentration during sealed incubations 
\citep[similar to][]{phillips2019}.
The incubations were conducted in situ at the respective sites to incorporate
spatial variation in ambient conditions, such as light and temperature.
Each mesocosm was first incubated under ambient light to give an
estimate of net ecosystem production (NEP),
followed by a dark incubation with the top of each mesocosm wrapped
in 4 layers of black plastic to give an estimate of ecosystem respiration (ER).
NEP + ER gives an estimate of GPP, assuming that ER is the same during both the light
and dark incubations. Half of the mesocosms at each site were incubated on 10 July,
while the other half were incubated on 11 July;
all of the mesocosms incubated on a given day for a given site were on the same 
experimental rack and so constituted a block.
The incubations lasted 3--5h, 
and the tops of the mesocosms were sealed with rubber stoppers for the duration.
DO was measured using a handheld probe (ProODO, YSI, Yellow Springs, Ohio, USA),
and we gently stirred the water within each mesocosm to homogenize it 
before taking the reading. 
We repeated the incubation procedure on 21 and 23 July (days 16 and 18).
Due to difficult weather, 
we were unable to perform the incubations at the respective sites.
Therefore, on 21 July all of the mesocosms were moved to a bay on the
southern shore of the south basin (depth $\approx$ 1.7m).
The light incubations lasted 3--5h,
while the dark incubations lasted 4--10h.
While variation in incubation duration was not ideal,
the DO in the dark incubations remained above anoxic conditions 
(minimum DO >10 $\text{mg L}^{-1}$). 
We converted GPP to units of $\text{mg}~\text{O}_2~\text{m}^{-2}~\text{h}^{-1}$,
accounting for incubation duration and water column depth within each mesocosm.

On 23 July, shortly before the expected time of midge emergence,
we removed the mesocosms from the lake and covered the top of each with mesh
to catch adult midges as they emerged. 
We kept the mesocosms outdoors in baths of cold tap water 
to moderate temperature.
Every 1--3d for the next 13d, we collected the emerging adults from the mesocosms.
While these were not individually identified, the vast majority appeared to be Tanytarsini.
Furthermore, there was a strong association between the number of Tanytarsini larvae 
stocked in the mesocosms and the number of adults that emerged 
(Spearman rank correlation of 0.82; $\emph{P}$ < 0.0001).

We quantified the relationship between GPP and larval density using a 
linear mixed model (LMM). 
The model included initial density (numeric), site (three levels),
incubation day (two levels; either days 5--6 or 16--18), 
and their two-way interactions as fixed effects.
Because we expected the relationship between GPP and initial density to be nonlinear,
we also included 2nd and 3rd order polynomial terms for initial density 
(without any interactions) in the model.
We chose a third degree polynomial because this gave the same number of parameters 
to estimate as would have been the case if each of the four density levels were treated 
categorically (including the intercept). 
The polynomial regression allowed us to treat density as a numeric variable 
and simplify the model by only including interactions with the linear density term.
We accounted for variation in ambient conditions during the incubations by including
linear terms for PAR and temperature estimated for each block at each site.
Finally, we included random effects for experimental rack and mesocosm identity to account
for blocking and repeated measures.

We used a binomial generalized linear mixed model (GLMM) to analyze variation 
in the number of midges that emerged as adults
relative to the initial number of larvae
(excluding the zero treatment). 
We included initial density (numeric), site (three levels), and their interaction
as fixed effects. 
We included random effects for experimental rack and mesocosm identity to account
for blocking and potential overdispersion, respectively;
the latter was equivalent to assuming the residuals followed a 
logit-normal binomial distribution.

Our goal in this study was to assess the effect of midge ecosystem engineering 
on their emergence.  
To do this, we fit a logit-normal binomial GLMM 
to the number of midges that emerged as adults
relative to the initial number of larvae 
with GPP per initial larva as the sole fixed effect. 
We then projected the number of emerging adult under two scenarios: 
(i) using predicted values of GPP as a function of site, incubation period, 
and larval density treatment according to the polynomial LMM described above, 
and (ii) using predicted values of GPP as a function of site and incubation period, 
but with larval density set to zero (on the natural scale). 
Scenario (ii) implies that GPP per larva declines across the midge treatments 
due purely to the greater number of larvae over which production is distributed. 
Scenario (i) differs from scenario (ii) by also including direct effects of 
larval density on GPP. 
The difference between scenarios (i) and (ii) gives a measure of the effect 
of larval density on the density dependence of survival and emergence 
with and without the effects of midge larvae on GPP.

Statistical analyses were conducted in R 4.0.3 \citep{r2020},
using the \texttt{lme4} package to fit the LMM and GLMMs \citep{lme4}.
We calculated $\emph{P}$-values with $\emph{F}$-tests using
the Kenward-Roger correction for the LMM
and parametric-bootstrapped likelihood-ratio tests (LRTs) based on 2000 simulations
for the GLMMs (\texttt{simulate} function in the native \texttt{stats} package).
We report both Type III and Type II tests unless otherwise noted,
to balance concerns of inflated Type I errors that can occur when dropping 
terms with the poor statistical inference that can come from overparameterized models.









