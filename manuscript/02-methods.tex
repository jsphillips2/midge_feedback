

\section*{Methods}

%========================================================================================

M\'{y}vatn is a large ($37\text{m}^2$), shallow (mean depth: 2.5m), 
naturally eutrophic lake in northeastern Iceland (65°40’N 17°00’W) \citep{einarsson2004}.
It is separated into two, ecologically distinct basins (north and south).
Our study was conducted at three sites (E2, E3, and E5) in the south basin 
that were selected to represent a range of ecological conditions (Figure \ref{fig:sites}).
These sites all have soft substrate that is suitble for \emph{T. gracilentus},
although they often differ in larval densities \citep{lindegaard1979}. 
However, in the summer of 2017 when we conducted our experiment, 
E2 was subject an expanding mat of filamentous green algae 
that was largely absent from E3 and E5 (J. Phillips; personal observation).
Furthermore, E2 was substantially colder during the experiment period than 
E3 and E5 (Figure \ref{fig:sites}).
In contrast, photosynthetically active radiation (PAR) was similar between the sites,
due to their similar depths (E2: 2.8m; E3: 3.3m; E5: 2.6m)
and high water clarity throughout the south basin.
Light and temperature data were collected with two loggers 
(HOBO Pendant, Onset Computer Corporation) deployed on the lake bottom at each site; 
PAR was recorded as visual light intensity (lux)
and converted to PAR using a standard scaling factor \citep{thimijan1983}.

In the summer of 2017, we conducted a field mesocosm experiment using a design
similar to \citep{phillips2019}.
On X, we collected sediment cores from the three study sites using a Kajak corer. 
For each site, we pooled the sediment from the different cores while keeping the 
top 5cm (``top'') and next 10cm (``bottom'' separate.
We then sieved the sediment through either 125 (top) or 500$\mu m$ (bottom) mesh
to remove midge larva.
The sediment was left to settle for X days in cool, dark, location.
We constructed the mesocosms by stocking the sediment into 
clear acrylic tubes (33cm height x 5cm diameter) 
sealed from the bottom with foam stoppers.
We added 10cm of bottom sediment and 5cm of top sediment, 
to mimic the sediment layering in the lake.
The sediment layer of each mesocosm was wrapped with 4 layers of black plasitc
to eliminate light from the sides of the mesocosms.

On X, we collected sediment cores from the lake and sieved them through 125 $\mu m$
to collect Tanytarsini larvae.
Tanytarsini progress through four instars before emerging as adults.
We attempted to select individuals the general size of second instar larvae
to maximize the duration of the experiment before emergence.
We then stocked the mescosoms with four densities of Tanytarsini larvae:
0, 50, 100, 200. 
This range of densities spans approximately half the range of densities observed in the lake.
We then filled the mesocosms with water collected from the southern shore of M\'{y}vatn's
south basin and gave the midges 24 h to settle before beploying in the lake.
On X, we distributed the mesocosms corrresponding to each site onto two racks and then
deployed them at their respective sites on the lake bottomo.







