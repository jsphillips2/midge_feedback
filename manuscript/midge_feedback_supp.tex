% ---------------------------------------------------------------------------------------
% ---------------------------------------------------------------------------------------
% Specifications
% ---------------------------------------------------------------------------------------
% ---------------------------------------------------------------------------------------
  
\documentclass[12pt]{article}
\usepackage[letterpaper, margin=1in]{geometry}
\usepackage{newtxtext,newtxmath}
\usepackage[math-style=ISO]{unicode-math}
\usepackage{fullpage}
\usepackage[authoryear,sectionbib]{natbib}
\linespread{1.7}
\usepackage[utf8]{inputenc}
\usepackage{lineno}
\usepackage{titlesec}
\titleformat{\section}[block]{\Large\bfseries\filcenter}{\thesection}{1em}{}
\titleformat{\subsection}[block]{\Large\itshape\filcenter}{\thesubsection}{1em}{}
\titleformat{\subsubsection}[block]{\large\itshape}{\thesubsubsection}{1em}{}
\titleformat{\paragraph}[runin]{\itshape}{\theparagraph}{1em}{}[. ]
\renewcommand{\refname}{Literature Cited}
\DeclareTextSymbolDefault{\dh}{T1}
\medmuskip=8mu 
\thickmuskip=10mu
\usepackage{graphicx}
\usepackage{booktabs}
\renewcommand{\thetable}{\Roman{table}}
\usepackage{caption}
\captionsetup{justification = raggedright,
singlelinecheck = false,
labelfont = bf}
\renewcommand{\thefigure}{S\arabic{figure}}
\renewcommand{\theequation}{S\arabic{equation}}
\renewcommand{\thetable}{S\arabic{table}}
\setcounter{equation}{0}
\setcounter{figure}{0}
\setcounter{table}{0}
\usepackage{amsmath}






% ---------------------------------------------------------------------------------------
% ---------------------------------------------------------------------------------------
% Title page
% ---------------------------------------------------------------------------------------
% ---------------------------------------------------------------------------------------
  
\title{\textbf{Supplementary materials:} \\ 
        Ecosystem engineering alters density-dependent feedbacks in an aquatic insect population}

\author{
  Joseph S. Phillips$^{1,2,\dagger}$ \\
  Amanda R. McCormick$^{1,3}$ \\
  Jamieson C. Botsch$^{1}$ \\
  Anthony R. Ives$^{1}$}

\usepackage{amsmath} % for split math environment

\date{}

\begin{document}

\raggedright
\setlength\parindent{0.25in}

\maketitle

\noindent{} 1. Department of Integrative Biology, University of Wisconsin, Madison, Wisconsin 53706 USA

\noindent{} 2. Department of Aquaculture and Fish Biology, H\'{o}lar University, Skagafj\"{o}r{\dh}ur 551 Iceland

\noindent{} 3. Department of Land, Air and Water Resources, 
University of California, Davis, CA, USA

\noindent{} $\dagger$ E-mail: joseph@holar.is

\linenumbers{}

\clearpage

% ---------------------------------------------------------------------------------------
% Map
% ---------------------------------------------------------------------------------------

\section*{Study sites}

Figure \ref{fig:sites}a shows the sites within M\'{y}vatn used for this study. 
Sediment was collected from three sites (E2, E3, and E5) for the establishment of the mesocosms,
while Tanytarsini larvae were all collected from E3. 
The mesocosms were deployed at their respective sediment-source sites, 
which is also where they were incubated for the first set of metabolism measurements
(either day 5 or 6). 
On day 16 they were moved to a common site for the final set of metabolism measurements,
as marked in Figure \ref{fig:sites}a.

Light and temperature data were collected with two loggers 
(HOBO Pendant, Onset Computer Corporation) deployed on the lake bottom at each site
and set to log every 30 minutes.
Light conditions were generally similar between the three sites (Figure \ref{fig:sites}b).
However, E2 was consistently colder than the other two sites.

\begin{figure}
\centering
\includegraphics{p_sites.pdf}
\caption{\label{fig:sites}
Experimental sites. 
(a) The three sites (E2, E3, and E5) covered a wide range of M\'{y}vatn's south basin.
Light gray areas indicate water, while white areas indicate land.
The black ``X'' marks the common site used for the final set of metabolism measurements.
(b) Mean daily PAR and temperature were calculated by averaging half-hourly
measurements from two loggers deployed on the lake bottom for each site.
}
\end{figure}

\clearpage

\section*{Mesocosms}

The experimental mesocosms were constructed from acrylic tubes filled with sediment
(Figure \ref{fig:photo1}),
as described in the main text.
The bottom portion of each mesocosm was filled with sieved lake sediment,
and top portion was filled with tap water in the lab.
The mesocosms were deployed on the lake bottom (Figure \ref{fig:photo2})
with their tops open to allow exchange with the surrounding water.
They were sealed with rubber stoppers (as shown in Figure \ref{fig:photo1})
during dissolved oxygen incubations for measuring GPP.


\clearpage

\begin{figure}
\centering
\includegraphics{mesocosms.pdf}
\caption{\label{fig:photo1}
Construction of experimental mesocosms.
}
\end{figure}

\begin{figure}
\centering
\includegraphics{underwater.pdf}
\caption{\label{fig:photo2}
Deployment of experimental mesocosms in the field.
}
\end{figure}

\clearpage

\section*{Statistical methods}

We used linear mixed models (LMMs) and generalized linear mixed models (GLMMs)
to analyze our data, using the \texttt{lme4} package \citep{lme4} 
in R 4.0.3 \citep{r2020}.
These analyses are described in the main text,
however we report the corresponding model formulas here for the sake of greater transparency.
Data and code for recreating these analyses are available on FigShare 
(link to be updated upon acceptance).

We analyzed responses of GPP to larval density using a 
polynomial LMM with the \texttt{lmer} function and the following model 
in R formula notation: 
%
\begin{multline} \label{eq:1}
\text{GPP} \sim \text{Temperature} + \text{PAR} + (\text{Day} + \text{Larvae} + \text{Site})^2 + \\ 
  \text{I}(\text{Larvae}^2) + \text{I}(\text{Larvae}^3) + (1|\text{Rack}/\text{Mesocosm}).
\end{multline}
%
Temperature and PAR are values from during the incubation period 
(as opposed to values during the deployment period, as shown in Figure \ref{fig:sites}b).
``Larvae'' refers to the initial number of Tanytarsini larvae stocked in the mesocosms,
``Rack'' refers to the deployment rack identity to accounting for blocking,
and ``Mesocosm'' refers to the mesocosm identity nested within Rack 
to account for repeated measurements of the same
mesocosms across the two sample days (Day 5/6 vs. Day 16/18).
Temperature, PAR, and Larvae were z-scored (subtracted mean and divided by standard deviation)
for model fitting.
Note that the superscript associated with the term 
$(\text{Trial} + \text{Larvae} + \text{Site})^2$,
denotes main effects and two-way interactions among the associated terms,
while the superscripts associated with the terms passed to the function ``I''
(e.g., $\text{I}(\text{Larvae}^2)$
denote exponentiation.

We quantified the relationship between emergence and larval density
using a binomial GLMM with the \texttt{glmer} function and the following model formula: 
%
\begin{equation} \label{eq:2}
\text{cbind}(\text{Emerged}, \text{Larvae} - \text{Emerged}) \sim \text{Larvae}*\text{Site} + 
        (1|\text{Rack}/\text{Mesocosm}).
\end{equation}
%
``Emerged'' refers to the number of individuals that emerged as adults,
which were modeled as ``successes'' from a number of binomial trials defined 
by the initial number of larvae stocked in the mesocosm.
The other variables are defined as above. 
``Larvae'' was z-scored on the right-hand side for the estimation
of the fixed effect, but left in raw form on the left-hand side 
for the calculation of the number of ``failed'' binomial trials (i.e., Larvae - Emerged).

We quantified the relationship between emergence and GPP per larva
using a binomial GLMM with the \texttt{glmer} function and the following model formula: 
%
\begin{equation} \label{eq:3}
\text{cbind}(\text{Emerged}, \text{Larvae} - \text{Emerged}) \sim \text{GPP per larva}  + 
          (1|\text{Rack}/\text{Mesocosm}).
\end{equation}
%
GPP per larva was calculated for each mesocosm from the observed GPP standardized to the mean
temperature using the temperature coefficient from equation \ref{eq:1}
and averaged between the two time points.
This value was then divided by the initial number of larvae and then z-scored.

To produce the curves in main text Figure 2,
we used the estimated coefficients from equation \ref{eq:1} to generate a range of GPPs for different 
sites and larval midge densities.
This was done either including the effect of larvae (scenario i) 
or by excluding this effect (scenario ii) as described in the main text.
Values were generated for both sample periods, which were then averaged.
We then divided these ``predicted'' values by the larval density to calculate 
the GPP per larva, as for the right-hand side of equation \ref{eq:3},
which we then used as the input to the model fit by equation \ref{eq:3} to generate
predicted proportions of midges emerged as a function of GPP per midge
under the two scenarios.
These projections were shown in main text Figure 2 both as proportions (panel a)
and as the total number of midges that would emerge under that scenario 
(panel b; calculated by multiplying the proportion by the initial number of larvae).

% ---------------------------------------------------------------------------------------
% ---------------------------------------------------------------------------------------
% Literature Cited
% ---------------------------------------------------------------------------------------
% ---------------------------------------------------------------------------------------


\bibliographystyle{ecology.bst}

\bibliography{refs.bib}

\clearpage

\end{document}
