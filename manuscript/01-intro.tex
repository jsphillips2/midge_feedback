\section*{Introduction}

Ecosystem engineering is a class of ecological interactions whereby 
one population affects others through alterations to the physical environment 
\citep{jones1994, wilby2002}.
Like all interspecific interactions, 
ecosystem engineering has the potential to generate feedbacks among various 
members of a community \citep{bertness1997,largaespada2012,donadi2014,sanders2014}.
For example, physical structure provided by coral can ameliorate competition with algae by
benefiting grazers that reduce algal abundance \citep{bozec2013}. 
As ecosystem engineers are (by definition) the source 
of engineering effects within an ecosystem, 
feedbacks between engineering and the engineers themselves are central 
to the dynamical consequences of engineering for the community as a whole 
\citep{hastings2007, sanders2014}. 

To understand the role of engineering feedbacks
for the population dynamics of ecosystem engineers,
it is useful to relate those feedbacks to the strength of density dependence
\citep{hastings2007, cuddington2009}.
Using a simple mathematical model,
Cuddington et al. (2009) showed that a wide range of dynamical behavior
is possible for populations of ecosystem engineers,
including stable persistence, extinction, unbounded growth, and alternative states.
The dependence of engineering effects on population density and 
the subsequent feedback of engineering to density dependence 
are key factors determining the overall dynamical consequences of ecosystem engineering.
Despite their theoretical importance,
quantitative characterizations of density-dependent 
engineering feedbacks for natural populations are limited.
While previous studies have established the existence of engineering-mediated feedbacks
to engineer populations \citep[e.g.,][]{largaespada2012, bozec2013, donadi2014},
they have generally not done so across a range of engineer densities 
as is required to directly quantify the density dependence of population growth.
 
We assessed the effect of ecosystem engineering on the sign and magnitude
of density-dependent survival and emergence of the midge \emph{Tanytarsus gracilentus} 
(Diptera: Chironomidae) in Lake M\'{y}vatn, Iceland. 
The larvae of \emph{T. gracilentus} dwell in the sediment and build silk tubes that 
elevate primary production by providing a substrate for algal growth 
\citep{herren2017, phillips2019},
similar to other aquatic macroinvertebrates 
\citep{largaespada2012,donadi2014,hoelker2015}.
The larvae feed on benthic algae (mainly diatoms), 
which means that their enhancement of benthic production may benefit their own 
survival, emergence, and subsequent reproduction \citep{ingvason2004}.
However, midge consumption may also reduce algal biomass,
potentially leading to intraspecific competition and negative density dependence 
\citep{einarsson2016}.
Indeed, the \emph{T. gracilentus} population 
in M\'{y}vatn shows large fluctuations in abundance 
that are likely driven by food limitation, although these fluctuations cannot be explained 
purely in terms of classical consumer-resource cycles \citep{ives2008}.
Characterizing the nature of density dependence is important for understanding 
the complex population dynamics of \emph{T. gracilentus},
making it a valuable case for exploring the effects of ecosystem engineering 
on density dependence of engineer populations.

To evaluate the role of ecosystem engineering 
on density-dependent survival and emergence in \emph{T. gracilentus},
we conducted a field mesocosm experiment across a range of experimental larval densities. 
This allowed us to directly quantify
(i) the relationship between benthic primary production and larval midge density and
(ii) the relationship between adult emergence rates and primary production per larval midge.
We then combined these two relationships with a simple model 
that allowed us to isolate the contribution of larval midge effects on 
primary production to their density-dependent survival and emergence.

