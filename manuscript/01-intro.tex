\section*{Introduction}

Ecosystem engineering is a class of ecological interactions whereby effects of one
population on another are mediated through alterations to the physical environment
\citep{jones1994, wilby2002}.
Like all interspecific interactions, 
ecosystem engineering has the potential to generate feedbacks among various 
members of a community \citep{bertness1997,largaespada2012,donadi2014,sanders2014}.
For example, physical structure provided by coral can ameliorate competition with algae by
benefiting grazers that reduce algal abundance \citep{bozec2013}. 
As ecosystem engineers are (by definition) the source of engineering effects within
an ecosystem, 
feedbacks between engineering and the engineers themselves
are central to the dynamical consequences of engineering for the 
community as a whole \citep{hastings2007, sanders2014}. 

To understand the role of engineering feedbacks
for the population dynamics of ecosystem engineers,
it is useful to relate those feedbacks to the strength of density dependence
\citep{hastings2007, cuddington2009}.
Using a simple mathematical model,
\cite{cuddington2009} showed that a wide range of dynamical behavior
is possible for populations of ecosystem engineers,
including stable persistence, extinction, unbounded growth, and alternative states.
Two key factors for determining such outcomes are 
(1) the dependence of engineering effect on population density and 
(2) the feedback of engineering to density dependence.
Despite their theoretical importance,
quantitative characterizations of such relationships for natural populations are limited.
While previous studies have established the existence of engineering-mediated feedbacks
to engineering populations \citep[e.g.,][]{bozec2013, donadi2014, largaespada2012}
they have generally not done so across a range of engineer densities 
as is required to directly quantify density dependence.
 
We assessed the effect of ecosystem engineering on the sign and magnitude
of density dependence in the midge \emph{Tanytarsus gracilenuts} (Diptera: Chironomidae)
in Lake M\'{y}vatn, Iceland. 
The larvae of \emph{T. gracilenuts} dwell in the sediment and build silk tubes that 
elevate primary production by providing a substrate for algal growth 
\citep{herren2017, phillips2019},
similar to other aquatic macroinvertbrates 
\citep{largaespada2012,donadi2014,hoelker2015}.
The larva feed on benthic algae (mainly diatoms), 
which means that their enhancement of benthic production may benefit their own 
survival and subsequent reproduction \citep{ingvason2004, einarsson2002}.
However, midge consumption may also reduce algal biomass,
potentially leading to intraspecific competition and negative density dependence.
\citep{einarsson2016}.
Indeed, M\'{y}vatn's \emph{T. gracilenuts} show large fluctuations 
in abundance that are likely driven by food limiation,
although these fluctuations cannot be explained purely in term classical 
consumer-resource cylces \citep{ives2008}.
Characterizing the nature of density dependence is important for understanding 
the complex population dynamics of \emph{T. gracilenuts},
making it a good example case for exploring the effects of ecosytem engineering 
on density dependence.

To evaluate the role of ecosystem engineering on density dependence in \emph{T. gracilenuts},
we conducted a field mesocosm experiment across a range of experimental larval densties. 
This allowed us to directly quantify
(a) the relationship between benthic primary produdction and larval midge density and
(b) the relationship between adult emergence rates and primary produdction per larval midge.
We then combined these two relationships into a simple model 
that allowed to isolate the contribution of larval midge effects on 
primary production to their density dependence.

