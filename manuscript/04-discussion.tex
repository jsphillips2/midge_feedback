
\section*{Discussion}

Midge effects ono GPP weakened negative density dependence,
particularly at intermedaite densities.

Thee nonlinear effect of midges on GPP likely resulted both from 
saturation of the beneficial effects of midges and reduction of biomass
through consumption.
Therefore, the effect of midge engineering was maximized at intermediate densities.
Our study only spanned half the range of ambient densities.
It is likely that at the highest densities, midge effects wouold be negative 
to do a depression of GPP (in addtion to scamble competition).

Grazing itself can elevate GPP by ameliorating competition, 
but has a tradeoff with biomass reduction.

Myvatn is an extreme case. 
But there are many aquatic inverts (some of which are harmful invasives)
that have similar engineering effects and may have similar dynamics.

GPP was similar across sites at the beginning of the experiment,
but quite dfferent at the end. 
Furthermore, the effect of midges varied across sites as well.
However, our analyses accounted for differences in ambient environmental conditions
during thte incubations. 
Therefore, this suggests that there was a legacy of environmental effects 
on GPP responses to midges.
While we do not have direct experimental demonstrationi this,
on candidate is temperature, which was consstently lweor at E2 
(where the mide effect was weakest) than E3 and E5.

While midge engneering weakend the negative densitiy dependence, 
it did not change the sign of density dependence. 
Furthermore, at the highest densities midge effects on GPP are likely
negatiive, making negative densitiy dependence even stronger.
This absence of positive density dependence has important implications 
for the role of eengineering in population dynamics.

Two important featurese identified by \cite{cuddington2009} that we did not examine are
(1) whether engineering is obligat and (2) the extent to which the engineering 
effect can persist in the absence of the engineer.
Detritus and organic material is abundnat and an important food source for midges,
so they may be able to survive in the absnece of their engineering effects. 
Fig 3 implies a y-interecept above 0, which is consitenet with this 
(although care is needed when extrapolating beyond the range of the data).
Tubes likely persist for some time after emergence, 
but likely not long enough to directly affect the next generation.
However, there may be subtler legacies.

Aquatic production and inverts: 
\citep{lamberti1983, wellnitz1996}
\citep{mayer1987, goldfinch2000, ingvason2004}

Legacies
\citep{cuddington2011}

Density depdence
\citep{turchin2003}