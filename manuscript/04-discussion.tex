
\section*{Discussion}

Our field mesocosm experiment shows how ecosystem engineering can weaken negative 
density dependence for the population of engineers.
We found that \emph{T. gracilentus} larvae in  M\'{y}vatn have large positive 
effects on benthic primary production,
which previous studies have shown are driven at least in part by physical 
structure provided by the tubes in which the larvae reside 
\citep{hoelker2015, phillips2019}.
Because \emph{T. gracilentus} larvae feed on benthic diatoms \citep{ingvason2004},
simulation of benthic production increased the amount of production per individual,
which in turn weakened negative density-dependence arising from food limitation.


While the effect off midges on benthic production was generally positive in our experiment,
this effect was nonlinear and plateaued between 60,000 and 90,000 
individuals $\text{m}^{-2}$.
Consequently, 
the midge-mediated weakening of density dependence was maximized
at intermediate densities.
Given the suppression of algal biomass through grazing \citep{einarsson2016},
it is likely that the effect of midges on production becomes negative
at the highest densites observed in the lake (>200,000 $\text{m}^{-2}$).
This suggests that while midges weaken negative density dependence at moderate
densities, they may enhance it at the highest densities.
The nonlinearity of ecosytem engineer effects has previously been identified 
as an important factor in governing the effects of engineering on community dynamics 
\citep{bozec2013}.
Despite weakening densitity dependence at low to moderate densities,
midge engineering did not lead to positive densitiy dependence 
\citep[i.e., allee effects;][]{courchamp1999}.
This has important implications for their population dynamics,
as positive density dependence could lead to run-away or overcompensatory growth
\citep{turchin2003, cuddington2009}.

The midge effect on benthic producivity across the three sites 
was similar at the beginning of the experiment,
but diverged through time even after accounting for variation in ambient conditions
during the productivity measurements.
This suggests that there was a legacy of local conditions that affected the 
response of benthic producers to midge engineering.
While our experiment was not directly able to test for such legacies,
a plausible candidate is temperature,
which was consistently lowest at the site with the weakest response to midges.
Various studies have identified the role of environmental variation in mediating 
the strength and sign of ecosystem engineering \citep{wright2006,lathlean2017}.
Enduring legacies of environemtnal mediation are particularly important,
as they may serve to decouple the dynamics of the engineers 
and their community-wide effects
\citep{cuddington2011}.
Such decoupling may in tern alter the nature of density-dependent feedbacks in 
systems with ecosystem engineering \citep{cuddington2009}.