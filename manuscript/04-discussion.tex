
\section*{Discussion}

Our field mesocosm experiment showed that an ecosystem engineer population 
has positive effects on its food resources, 
thereby establishing a positive feedback 
that alters density-dependent survival and emergence. 
We found that Tanytarsini larvae in  M\'{y}vatn have large positive 
effects on benthic primary production,
which previous studies have shown are driven at least in part by physical 
structure provided by the silk tubes in which the larvae reside 
\citep{hoelker2015, phillips2019}.
Because midge larvae feed on benthic diatoms \citep{ingvason2004},
stimulation of benthic production increased the amount of food per individual midge,
which in turn weakened negative density-dependence arising from food limitation
at moderate midge densities.
However, the amelioration of negative density dependence largely disappeared at high densities,
presumably due to midge suppression of diatom growth through consumption.

The nonlinear feedback of midge engineering on primary production meant 
that the weakening of density dependence in emergence was greatest at intermediate densities. 
Given the suppression of algal biomass through grazing \citep{einarsson2016},
it is likely that the effect of midges on production becomes negative
at the highest densities observed in the lake.
This suggests that while the feedback through midge engineering weakens 
negative density dependence at moderate densities, 
this is not enough to overcome the consumptive effect of midges at high densities. 
The nonlinearity of ecosystem engineer effects has previously been identified 
as an important factor in governing the effects of engineering on community dynamics 
\citep{bozec2013}.
Despite weakening density dependence at low to moderate densities,
midge engineering did not lead to positive density dependence in per capita emergence.
However, the nonlinear engineering feedback did result in a hump-shaped relationship
between total emergence and larval density, which could lead to overcompensatory dynamics
\citep{turchin2003, cuddington2009}.
It is important to note that our study does not include information on reproduction
and so does not fully capture the effect of density  dependence on the full life cycle.
Nonetheless, given the short duration of the adult stage and high egg production per adult 
\emph{T. gracilentus}, 
it is possible that the dynamics of the ``engineering stage'' (i.e., larvae)
are principally driven by larval survival, of which adult emergence is a direct extension.
While ecosystem-engineering effects of other benthic invertebrates may differ from 
tube-building midges by being concentrated in the adult stage 
\citep[e.g. mussels;][]{largaespada2012},
juvenile production may be similarly unlimited by adult abundance such that 
survival of the engineering stage is most relevant for their dynamics.

The midge effect on benthic productivity across the three sites 
was similar at the beginning of the experiment,
but diverged through time even after accounting for variation in ambient conditions
during the productivity measurements.
Furthermore, the engineering feedback subtly differed between sites,
with midge emergence experiencing the greatest benefit from engineering at E3,
which was the site with the greatest ambient densities at the beginning of the experiment.
This suggests that local environmental context altered the consequences 
of ecosystem engineering, 
and these alterations had the potential to persist through time.
While our experiment was not directly able to test for such legacies,
plausible candidate causes are temperature (which was consistently lower at E2),
and sediment nutrient concentrations which vary across locations.
These environmental variables could have altered algal 
community composition or abundance
\citep{gudmundsdottir2011, mccormick2019},
thereby mediating their capacity to respond to the 
amelioration of light limitation provided by midge engineering
\citep{phillips2019}.
Various studies have identified the role of environmental variation in mediating 
the strength and sign of ecosystem engineering 
\citep{wright2006,lathlean2017}.
When the effects of environmental mediation persist through time,
they may decouple the dynamics of the engineers and their community-wide effects.
Such decoupling fundamentally alters the density-dependence of engineering feedbacks,
and therefore may play a central role in determining the dynamics of ecosystem engineering
\citep{cuddington2009}.